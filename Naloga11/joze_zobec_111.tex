\documentclass[12pt, a4paper]{article}
\usepackage[slovene]{babel}
\usepackage[utf8]{inputenc}
\usepackage[T1]{fontenc}
\usepackage{amsmath, amssymb, fullpage}

\begin{document}

\section{Uvod}

Pri tej nalogi smo imeli nekaj dela s teorijo grafov. Natan\v{c}neje, opravka smo imeli z obojestransko povezanimi
grafi. Z njimi lahko namre\v{c} natan\v{c}neje obravnavamo populacijske modele, saj so taki grafi lokalizirani.

Modeliramo lahko na primer prena\v{s}anje vica/ideje/bolezni preko populacije. Povezave med ljudmi v taki populaciji so
stohasti\v{c}ne.

Poleg tega je bilo potrebno \v{s}e preveriti Kapitzov populacijski model. Kot pi\v{s}e v navodilu je pri\v{c}el s
preprostimi predpostavkami in pri\v{s}el do zanimivih rezultatov. To bomo obravnavali v nalogah 2 in 3.

\section{Povezave}

V tej nalogi smo pri\v celi z grafom, kjer je bilo vsako vozli\v s\v ce povezano zgolj z najblji\v zjimi sosedi. To
\v stevilo mora biti sodo, saj bi v nasprotnem primeru dobili orientirane grafe. Prav tako moramo predpostaviti, da je
vsako vozli\v s\v ce povezano samo s seboj. Tako lahko s potenciranjem matrike povezav pridemo do premera grafa.

Tak graf seveda ni realen -- da dobimo populacijo, ki nas bolje opi\v se moramo povezave stohasti\v cno premikati
znotraj matrike povezav. Ta je simetri\v cna, ima zgolj enice in ni\v cle. Indeksi stolpcev predstavljajo vozli\v s\v ca,
indeksi vrstic pa predstavljajo vozli\v s\v ce, na katerega je stolpec povezan.

Ker poznamo \v stevilo povezav $D$, ki je majhno v primerjavi s \v stevilom vozli\v s\v c, je veliko prostora za
optimizacijo. Sam sem na primer vse povezave pospravil v lo\v ceno tabelo, naklju\v cno \v stevilo, pa je predstavljalo
indeks povezave. S tem se izognemo nepotrebnem \v zrebanju \v stevil. Dobljeno enico v matriki pobri\v semo in prestavimo
na novo mesto. Tu je spet prostor za optimizacijo: enic je veliko manj kot ni\v cel, zato se teh ne spla\v ca imeti
pospravljenih v lo\v ceni tabeli. Zato sem najprej preveril, \v ce je verjetnost, da iz\v zrebamo ni\v clo ve\v cja od
0.5. \v Ce je, potem lahko \v zrebamo, sicer pa jo enostavno premikako po sosedih, dokler ne pridemo do nje in jo
pretvorimo v enico.

To po\v cnemo, dokler so izpolnjeni vsi izmed slede\v cih pogojev:
\begin{itemize}
	\item[(a)]{\v stevilo iteracij manj\v se od $10^4$,}
	\item[(b)]{koeficient gru\v cavosti je ve\v cji od $0.001$,}
	\item[(c)]{gru\v cavost se je zadnjih $100$ iteracij spremenila za ve\v c kot $0.001$.}
\end{itemize}

\end{document}
