\documentclass[a4 paper, 12pt]{article}
\usepackage[slovene]{babel}
\usepackage[utf8]{inputenc}
\usepackage[T1]{fontenc}
\usepackage[small]{caption}
\usepackage[pdftex]{graphicx}
\usepackage{amssymb, fullpage, float, pdflscape, subcaption, amsmath, color}

\captionsetup{
	size=small,
	labelfont=it,
	width=0.7\textwidth
}

\newcommand{\tr}{
	\operatorname{tr}
}

\begin{document}

\begin{center}
\textsc{Modelska analiza I}\\
\textsc{2011/12}\\[0.5cm]
\textbf{11. naloga -- Izbolj\v sani populacijski modeli}
\end{center}
\begin{flushright}
\textbf{Jože Zobec}\\
\end{flushright}

\section{Uvod}

V tem poglavju se bomo spet posvetili odbiranju parametrov s pomo\v cjo metode najmanj\v sih kvadratov in
modeliranja populacijskih modelov s krajevno odvisnostjo, ali druga\v ce povedano, grafov.

Algoritem, ki ga bomo uporabili za modeliranje, sta prva uporabila Watts in Strogatz in ni popolnoma realisti\v cen,
vendar pa je dovolj dober, da ga lahko u\v cinkovito uporabimo modeliranje manj\v sih skupin. Za\v cnemo s
s periodi\v cnim neorientiranim grafom z $N$ elementi, ki je povezan z $D$ sosedi (tj. $D/2$ na vsako stran). Tak
graf lahko predstavimo z matriko povezav, $M$ in ker je graf neorientiran (povezave te\v cejo v obe smeri) je matrika
simetri\v cna. Da lahko pravilno ra\v cunamo premer grafa, bomo predpostavili, da je vsak povezan sam s seboj,
tj. velja $\tr(M) = N$, kjer smo vsako med $i$ in $j$ v matriki ozna\v cili z $M_{ij} = M_{ji} = 1$.

Kot je povedal profesor na predavanjih ima tak graf predvsem dva neodvisna parametra -- povpre\v cno razdaljo poti in
gru\v cavost grafa. A ker je povpre\v cno pot zahtevno ra\v cunati, jo bomo nadomestili s premerom grafa, to je
najdalj\v sa pot v grafu. Premer grafa je tista potenca $r$ matrike $M$, pri kateri je $M_{ij} > 0,\ \forall\ i,j$.
Za povpre\v cno razdaljo bi morali pri vsaki potenci pogledati koliko elementov matrike je postalo neni\v celnih -- tisto
so nove povezave, dol\v zine, ki je eneka potenci te matrike $M$.

Gru\v cavost meri koliko sosedov nekega vozli\v s\v ca ostane sosedov, \v ce tisto vozli\v s\v ce odmislimo, povpre\v ceno
po vseh vozli\v s\v cih.

V drugem delu se bomo seznanili spet s prilagajanjem krivulj, tokrat so to krivulje, ki so samopodobne. S tujko
jih lahko imenujemo tudi renormalizabilne krivulje. Renormalizabilnost je lastnost sistemov, ki so neodvisni od
skale na kateri jih opazujemo. Tak primer krivulje je eksponentna funkcija.

\section{Rezultati}

Deloma sem programiranje opravil v {\tt Octave}, deloma v programskem jeziku {\tt C}, da bi dosegel optimalno
razmerje med hitrostjo delovanja in nazornostjo programa/skripte.

\subsection{Model mali svet}

Pred kratkim se je v Kranju vr\v silo sre\v canje oktetov, ki je imelo $\sim 320$ poslu\v salcev. Mislimo si, da
je nekdo sredi nastopa Okteta Gallus iz Ribnice imel hodomu\v sno pripombo. Poglejmo, kako hitro bi ta novica pri\v sla
do njihovih u\v ses.

Pri\v cnemo torej z matriko povezav $N = 320$ in $D = 8$, saj je oktet spremljal tudi njihov umetni\v ski vodja, ki
z njimi ni nastopal -- to je devet ljudi. Sedaj moramo povezave stohasti\v cno premikati, kar prikazujeta grafa~\ref{gr1}
in~\ref{gr2}. \v Stevilo stohasti\v cnih povezav sem ra\v cunal z oceno, da vsaki\v c izberemo drugo povezavo, ki \v se
ni bila izbrana. \v Stevilo vseh povezav je $T = ND/2$, torej je $\sigma = \kappa \cdot 2/ND$, kjer je $\kappa$ \v stevilo
stohasti\v cno zamenjanih povezav.
\begin{figure}[H]\centering
	\input{plot1.tex}
	\caption{Premer grafa zelo hitro pade -- mimimum dose\v ze \v ze ko smo \v sele eno desetino povezav.}
	\label{gr1}
\end{figure}
\begin{figure}[H]\centering
	\input{plot2.tex}
	\caption{Gru\v cavost pada dosti po\v casneje.}
	\label{gr2}
\end{figure}

V vsakem koraku je verjetnost, da se bosta soseda pogovarjala o tej pikolovski pripombi je $p = 10\%$.
\begin{figure}[H]\centering
	\input{joke.tex}
	\caption{V primeru, ko povezav ne preme\v samo, dobimo linerano \v sirjenje informacije. Sicer
		pa \v ze za zelo majhno \v stevilo stohasti\v cno zamenjanih povezav dobimo ti. logisti\v cno funkcijo.
		Vidimo, da se hitrost \v sirjenja govorice prakti\v cno ne spreminja, ko dose\v zemo $\sigma \sim 0.6$.}
	\label{gr3}
\end{figure}

\end{document}
