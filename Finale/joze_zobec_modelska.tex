\documentclass[12pt, a4paper]{article}
\usepackage[slovene]{babel}
\usepackage[utf8]{inputenc}
\usepackage[T1]{fontenc}
\usepackage{amsmath, amssymb}

\renewcommand{\r}{
    \ensuremath{\mathbf{r}}
}

\newcommand{\p}{
    \ensuremath{\mathbf{p}}
}

\renewcommand{\L}{
    \ensuremath{\mathcal{L}}
}

\newcommand{\e}{
    \ensuremath{\mathrm{e}}
}

\newcommand{\sfrac}[2]{
    \ensuremath{\textstyle{\frac{#1}{#2}}}
}

\newcommand{\Frac}[2]{
    \ensuremath{\displaystyle{\frac{#1}{#2}}}
}

\newcommand{\der}[3][]{
    \ensuremath{ \frac{\partial^{#1} #2}{\partial #3^{#1}} }
}

\newcommand{\pl}{
    \ensuremath{\dot{+}}
}

\renewcommand{\d}{
    \ensuremath{\mathrm{d}}
}

\newcommand{\w}{
    \ensuremath{\omega}
}

\begin{document}

\section{Navodilo}

Razi\v s\v ci prostor kro\v znih orbit planeta okrog dvojne zvezde v pribli\v zku zanemarljive mase
planeta. Koliko parametrov ima model?

\section{Razmislek}

V pribli\v zku zanemarljiva mase planeta, je vsa dinamika ki jo potrebujemo vezana na gibanje planeta
v \v casovno odvisnem zunanjem potencialu (ki ga ustvarjata zvezdi) $V_\odot(\r, t)$. Hamiltonian celotnega
sistema je
\begin{equation}
    H = \frac{1}{2}\sum_{k = 1}^N \frac{|\p_k|^2}{m_k} - \frac{\kappa}{4\pi}\sum_{k,\ell}
        \frac{m_k m_\ell}{|\r_k - \r_\ell|},
    \label{hamilton}
\end{equation}
vektorje smo ozna\v cili z mastnim tiskom.

\subsection{Prosti parametri}

Da preverimo \v stevilo prostih parametrov moramo hamiltonko~\eqref{hamilton} pretvoriti v brezdimenzijsko
obliko. Energija se bo \v zal ohranjala samo, kadar upo\v stevamo tudi maso planeta, tako da je ne bomo
zanemarili. Naj bo $m_3$ masa planeta, $m_1$ in $m_2$ pa naj bosta masi zvezde. Definirajmo brezdimenzijski
masi
\begin{equation}
    M_1 \equiv \frac{m_1}{m_3}, \quad M_2 \equiv \frac{m_2}{m_3}.
\end{equation}
Nekaj svobode imamo \v se za to, kako bomo merili dol\v zine. Smiselno je, da bi merili v enotah oddaljenosti
zvezd na za\v cetku, tj. $\rho_0 = |\r_1 (0) - \r_2 (0)|$. Da bomo lahko naredili polno brezdimenzijsko
transformacijo potrebujemo \v se skalo za \v cas. Merili ga bomo v enotah $\tau$, ki je
\[
    \frac{1}{\tau^2} \equiv \frac{\kappa m_3}{4\pi\rho_0^3}.
\]
Hamiltonian se potem glasi
\begin{align*}
    H =& \frac{\rho_0^2}{2 m_3\tau^2} \big(\sfrac{1}{M_1}|\p_1|^2\tau^2/\rho_0^2 + \sfrac{1}{M_2}|\p_2|^2\tau^2/\rho_0^2 +
        |\p_3|^2\tau^2/\rho_0^2\big) -\\
       &- \Frac{\kappa m_3^2}{4\pi\rho_0}\bigg(\Frac{M_1}{|\r_1 - \r_3|/\rho_0} + \Frac{M_2}{|\r_2 - \r_3|/\rho_0} +
        \Frac{M_1 M_2}{|\r_1 - \r_2|/\rho_0}\bigg),
\end{align*}
kjer lahko konstanto $\kappa m_3^2/4\pi\rho_0$ izrazimo s $\tau$ in dobimo $m_3\rho_0^2/\tau^2$.
Sedaj uvedemo brezdimenzijske impulze in brezdimenzijske dol\v zine
\begin{equation}
    \frac{\p}{m_3\rho_0/\tau} \mapsto \p, \quad \frac{\r}{\rho_0} \mapsto \r,
\end{equation}
kar hamiltonian poenostavi v
\begin{align*}
    H =& \frac{m_3\rho_0^2}{2\tau^2} (\sfrac{1}{M_1}|\p_1|^2 + \sfrac{1}{M_2}|\p_2|^2 + |\p_3|^2) -\\
       &- \Frac{m_3\rho_0^2}{\tau^2}\bigg(\Frac{M_1}{|\r_1 - \r_3|} + \Frac{M_2}{|\r_2 - \r_3|} +
        \Frac{M_1 M_2}{|\r_1 - \r_2|}\bigg).
\end{align*}
Kar nam \v se preostana je, da uvedemo brezdimenzijsko transformacijo in energijo merimo v enotah
$m_3\rho_0^2/\tau^2$, tj.
\[
    \frac{\tau^2}{\rho_0^2 m_3}H \mapsto H
\]
in dobimo brezdimenzijski hamiltonian
\begin{equation}
    H = \frac{1}{2}\Big(\sfrac{1}{M_1}|\p_1|^2 + \sfrac{1}{M_2}|\p_2|^2 + |\p_3|^2\Big)
        - \bigg(\Frac{M_1}{|\r_1 - \r_3|} + \Frac{M_2}{|\r_2 - \r_3|} + \Frac{M_1 M_2}{|\r_1 - \r_2|}\bigg).
    \label{brezdim}
\end{equation}
Prosti parametri modela so $M_1$, $M_2$ in pa za\v cetni pogoji, ki jih je $2 \cdot 3d$, kjer je $d$
\v stevilo prostorskih dimenzij, ki so nam na voljo. \v Ce se dr\v zimo vezi, da je oddaljenost med zvezdama
na za\v cetku $1$, imamo en za\v cetni pogoj manj\footnote{Ena\v cba~\eqref{brezdim} velja tudi \v ce je
$\rho_0$ karkoli drugega.}.

Ta hamiltonian je \v cisto splo\v sen za gibanje treh (ne nujno nebesnih) to\v ckastih teles pod vplivom sile
gravitacije. Da bodo zvezde res zvezde, si bomo za zgled vzeli res te\v zak planet, Jupiter, ki je \v se
vedno tiso\v c-krat la\v zji od na\v sega Sonca. Kot smiselna se zato zdi omejitev $M_{1,2} \geq 1000$.
Prav tako je smiselno, da so razdalje med zvezdama na za\v cetku res 1.

\section{Pribli\v zek zanemarljive mase planeta}

Hamiltonian~\eqref{brezdim} do sedaj eksaktno ohranja energijo. Vendar vidimo, da bo \v clen v potencialu, ki je
sorazmeren z $M_1 M_2$ bistveno ve\v cji od \v clenov, ki so sorazmerni zgolj z $M_{1,2}$ (mogo\v ce za faktor 1000),
saj sta $M_1$ in $M_2$ vsaj reda velikosti $\mathcal{O}(10^3)$. To pomeni, da planet ne bo bistveno vplival na
spremembo gibalne koli\v cine zvezde,
\begin{equation}
    \d\p_{1,2} = \dot{\p}_{1,2}\ \d t = \{\p_{1,2}, V\}\ \d t \approx
        \Big\{\p_{1,2}, -\sfrac{M_1M_2}{|\r_1 - \r_2|}\Big\}\ \d t
\end{equation}
Prav tako se zvezdi glede na planet premikata zelo po\v casi -- efekt planete bi bil viden \v sele po zelo dolgem
\v casu (po mnogih planetarnih obhodih), saj je sprememba $\r$ ute\v zena z $1/M_{1,2}$. Sprememba koordinat
je
\begin{equation}
    \d \r_{1,2} = \dot{\r}_{1,2}\ \d t = \{\r_{1,2}, T\}\ \d t = \big\{\r_{1,2}, \sfrac{1}{2M_{1,2}}|\p_{1,2}|^2\big\}\ \d t.
\end{equation}
Ker smo pri spremembi gibalne koli\v cine naredili napako, ki je bila reda $\mathcal{O}(10^3)$, je ta napaka sedaj
samo $\mathcal{O}(10^{-6})$, torej smo vpliv planeta upravi\v ceno zanemarili.
\subsection{Orbiti zvezd}
Orbiti zvezd, $\r_1(t)$ in $\r_2(t)$, v pribli\v zku zanemarljive mase planeta lahko, izra\v cunamo analiti\v cno brez
uporabe numeri\v cnih orodij. Iz Poissonovih oklepajev dobimo slede\v ce gibalne ena\v cbe:
\begin{align}
    \dot{\r}_{1,2}(t) &= \frac{\p_{1,2}(t)}{M_{1,2}} \\
    \dot{\p}_{1,2}(t) &= \nabla_{1,2} \frac{M_1 M_2}{|\r_1 - \r_2|} = \pm \frac{M_1 M_2}{|\r_1 - \r_2|^3}(\r_2 - \r_1)
    \label{predgibalni}
\end{align}
Ker zvezdi ne \v cutita vpliva planeta, gremo lahko v sistem te\v zi\v s\v ca zvezd in uvedemo
\begin{align*}
    \r_c &= \sfrac{1}{2}(\r_1 + \r_2), \\
    \r_* &= \sfrac{1}{2}(\r_1 - \r_2),
\end{align*}
kjer je $\r_c$ seveda te\v zi\v s\v ce dvozvezdnega sistema, $\r_*$ pa je vektor, katerega dvakratnik nas iz prve
zvezde pripelje v drugo. Vektorja $\r_1$ in $\r_2$ lahko zapi\v semo kot linearni kombinaciji $\r_c$ in $\r_*$,
\begin{align*}
    \r_1 &= \r_c + \r_*, \\
    \r_2 &= \r_c - \r_*.
\end{align*}
Vektorjema $\r_c$ in $\r_*$ ustrezata gibalni koli\v cini $\p_c$ in $\p_*$. Dobimo ju iz gibalnih ena\v cb za $\r_1$
in $\r_2$:
\begin{align}
    \dot{\p}_c &= \p_1 + \p_2, \notag \\
    \dot{\p}_* &= \p_1 - \p_2,
\end{align}
kjer vidimo, da mora biti $\p_c$ konstanta gibanja, \v ce je izpeljava pravilna. Gibalni ena\v cbi za $\r_c$ in $\r_*$ se
potem glasita
\begin{align}
    \dot{\r}_c &= (\dot{\r}_1 + \dot{\r}_2)/2 = \sfrac{1}{2}\p_c\big(\sfrac{1}{M_1} + \sfrac{1}{M_2}\big) +
        \sfrac{1}{2}\p_*\big(\sfrac{1}{M_1} - \sfrac{1}{M_2}\big) \notag \\
    \dot{\r}_* &= (\dot{\r}_1 - \dot{\r}_2)/2 = \sfrac{1}{2}\p_*\big(\sfrac{1}{M_1} + \sfrac{1}{M_2}\big) +
        \sfrac{1}{2}\p_c\big(\sfrac{1}{M_1} - \sfrac{1}{M_2}\big) \notag \\
\end{align}
gibalni ena\v cbi za $\p_c$ in $\p_*$ pa dobimo iz
\begin{align}
    \p_1 &= \sfrac{1}{2}(\p_c + \p_*), \notag \\
    \p_2 &= \sfrac{1}{2}(\p_c - \p_*).
\end{align}
To lahko vstavimo v ena\v cbi~\eqref{predgibalni}, kar $\p_1$ in $\p_2$ izrazi v obliki
\begin{align*}
    \dot{\p}_1 &= (\dot{\p}_c + \dot{\p}_*)/2 = \frac{M_1M_2}{|2\r_*|^3}(-2\r_*) = -\frac{M_1M_2}{4|\r_*|^3}\r_*\ , \\
    \dot{\p}_2 &= (\dot{\p}_c - \dot{\p}_*)/2 = \frac{M_1M_2}{|2\r_*|^3}\cdot 2\r_* = \frac{M_1M_2}{4|\r_*|^3}\r_*\ .
\end{align*}
\v Ce ti dve ena\v cbi se\v stejemo, nemudoma dobimo $\dot{\p}_c = 0$, kar smo tudi pri\v cakovali. Nadaljujemo z
razliko prej\v snjih ena\v cb, po katerih dobimo $\dot{\p}_*$,
\begin{equation*}
    \dot{\p}_* = - \frac{M_1M_2}{2|\r_*|^3}\r_*.
\end{equation*}
Sedaj imamo dve opciji: lahko centriramo impulz, ali pa te\v zi\v s\v ce. Zavoljo nadaljnega numeri\v cnega re\v sevanja
problema, je bolje centrirati te\v zi\v s\v ce, tj. zahtevamo $\dot{\r}_c = 0$ in $\r_c (t = 0) = 0$. To mo\v cno
poenostavi vse skupaj, saj potem $\r_1 = \r_*$ in $\r_2 = -\r_*$. Newtonov zakon za $\r_*$ se glasi
\begin{align}
    \ddot{\r}_* &= \sfrac{1}{2}\dot{\p}_*\big(\sfrac{1}{M_1} + \sfrac{1}{M_2}\big) \notag \\
                &= -\frac{M_1 M_2}{2}\bigg(\frac{1}{M_1} + \frac{1}{M_2}\bigg)\frac{1}{|\r_*|^3}\r_* \notag \\
                &= -\frac{M_1 + M_2}{2|\r_*|^2}\ \hat{\r}_*\ ,
    \label{newton}
\end{align}
kjer je $\hat{\r}_*$ enostski vektor v smeri $\r_*$. Ta problem je dvo-dimenzionalen, uporabimo lahko nastavek
\[
    \r_* = \begin{bmatrix} \rho(t) \cos(\w t) \\ \rho(t) \sin(\w t) \end{bmatrix}, \quad |\r_*(t)| = \rho(t),
\]
ga vstavimo v ena\v cbo~\eqref{newton} in dobimo
\[
    \ddot{\rho}(t) - \w^2\rho(t) - \frac{\Delta M}{2}\frac{1}{\rho^2(t)} = 0.
\]



\end{document}
