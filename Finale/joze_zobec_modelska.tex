\documentclass[12pt, a4paper]{article}
\usepackage[slovene]{babel}
\usepackage[utf8]{inputenc}
\usepackage[T1]{fontenc}
\usepackage{amsmath, amssymb, fullpage, graphicx, float}

\renewcommand{\r}{
    \ensuremath{\mathbf{r}}
}

\newcommand{\p}{
    \ensuremath{\mathbf{p}}
}

\renewcommand{\L}{
    \ensuremath{\mathcal{L}}
}

\newcommand{\e}{
    \ensuremath{\mathrm{e}}
}

\newcommand{\ee}{
    \ensuremath{\mathbf{e}}
}

\newcommand{\sfrac}[2]{
    \ensuremath{\textstyle{\frac{#1}{#2}}}
}

\newcommand{\Frac}[2]{
    \ensuremath{\displaystyle{\frac{#1}{#2}}}
}

\newcommand{\der}[3][]{
    \ensuremath{ \frac{\partial^{#1} #2}{\partial #3^{#1}} }
}

\newcommand{\pl}{
    \ensuremath{\dot{+}}
}

\renewcommand{\d}{
    \ensuremath{\mathrm{d}}
}

\newcommand{\w}{
    \ensuremath{\omega}
}

\newcommand{\Varphi}{
    \ensuremath{\tilde{\varphi}}
}

\newcommand{\Vef}{
    \ensuremath{V_\mathrm{ef}}
}

\newcommand{\avgt}[1]{
	\ensuremath{\left\langle #1 \right\rangle_t}
}

\newcommand{\then}{
	\ensuremath{\Rightarrow}
}

\begin{document}

\section{Navodilo}

Razi\v s\v ci prostor kro\v znih orbit planeta okrog dvojne zvezde v pribli\v zku zanemarljive mase
planeta. Koliko parametrov ima model?

\section{Razmislek}

V pribli\v zku zanemarljive mase planeta, je vsa dinamika ki jo potrebujemo vezana na gibanje planeta
v \v casovno odvisnem zunanjem potencialu (ki ga ustvarjata zvezdi) $V_\odot(\r, t)$. Hamiltonian celotnega
sistema je
\begin{equation}
    H = \frac{1}{2}\sum_{k = 1}^N \frac{|\p_k|^2}{m_k} - \frac{\kappa}{4\pi}\sum_{k,\ell}
        \frac{m_k m_\ell}{|\r_k - \r_\ell|},
    \label{hamilton}
\end{equation}
vektorje smo ozna\v cili z mastnim tiskom.

\subsection{Prosti parametri}

Da preverimo \v stevilo prostih parametrov moramo hamiltonko~\eqref{hamilton} pretvoriti v brezdimenzijsko
obliko. Naj bo $m_3$ masa planeta, $m_1$ in $m_2$ pa naj bosta masi zvezde. Definirajmo brezdimenzijski
masi
\begin{equation}
    M_1 \equiv \frac{m_1}{m_3}, \quad M_2 \equiv \frac{m_2}{m_3}.
\end{equation}
Nekaj svobode imamo \v se za to, kako bomo merili dol\v zine. Smiselno je, da bi merili v enotah oddaljenosti
zvezd na za\v cetku, tj. $\rho_0 = |\r_1 (0) - \r_2 (0)|$. Da bomo lahko naredili polno brezdimenzijsko
transformacijo potrebujemo \v se skalo za \v cas. Merili ga bomo v enotah $\tau$, ki je
\[
    \frac{1}{\tau^2} \equiv \frac{\kappa m_3}{4\pi\rho_0^3}.
\]
Hamiltonian se potem glasi
\begin{align*}
    H =& \frac{\rho_0^2}{2 m_3\tau^2} \big(\sfrac{1}{M_1}|\p_1|^2\tau^2/\rho_0^2 + \sfrac{1}{M_2}|\p_2|^2\tau^2/\rho_0^2 +
        |\p_3|^2\tau^2/\rho_0^2\big) -\\
       &- \Frac{\kappa m_3^2}{4\pi\rho_0}\bigg(\Frac{M_1}{|\r_1 - \r_3|/\rho_0} + \Frac{M_2}{|\r_2 - \r_3|/\rho_0} +
        \Frac{M_1 M_2}{|\r_1 - \r_2|/\rho_0}\bigg),
\end{align*}
kjer lahko konstanto $\kappa m_3^2/4\pi\rho_0$ izrazimo s $\tau$ in dobimo $m_3\rho_0^2/\tau^2$.
Sedaj uvedemo brezdimenzijske impulze in brezdimenzijske dol\v zine
\begin{equation}
    \frac{\p}{m_3\rho_0/\tau} \mapsto \p, \quad \frac{\r}{\rho_0} \mapsto \r,
\end{equation}
kar hamiltonian poenostavi v
\begin{align*}
    H =& \frac{m_3\rho_0^2}{2\tau^2} (\sfrac{1}{M_1}|\p_1|^2 + \sfrac{1}{M_2}|\p_2|^2 + |\p_3|^2) -\\
       &- \Frac{m_3\rho_0^2}{\tau^2}\bigg(\Frac{M_1}{|\r_1 - \r_3|} + \Frac{M_2}{|\r_2 - \r_3|} +
        \Frac{M_1 M_2}{|\r_1 - \r_2|}\bigg).
\end{align*}
Kar nam \v se preostana je, da uvedemo brezdimenzijsko transformacijo in energijo merimo v enotah
$m_3\rho_0^2/\tau^2$, tj.
\[
    \frac{\tau^2}{\rho_0^2 m_3}H \mapsto H
\]
in dobimo brezdimenzijski hamiltonian
\begin{equation}
    H = \frac{1}{2}\Big(\sfrac{1}{M_1}|\p_1|^2 + \sfrac{1}{M_2}|\p_2|^2 + |\p_3|^2\Big)
        - \bigg(\Frac{M_1}{|\r_1 - \r_3|} + \Frac{M_2}{|\r_2 - \r_3|} + \Frac{M_1 M_2}{|\r_1 - \r_2|}\bigg).
    \label{brezdim}
\end{equation}
Prosti parametri modela so $M_1$, $M_2$ in pa za\v cetni pogoji, ki jih je $2 \cdot 3d$, kjer je $d$
\v stevilo prostorskih dimenzij, ki so nam na voljo. \v Ce se dr\v zimo vezi, da je oddaljenost med zvezdama
na za\v cetku $1$, imamo en za\v cetni pogoj manj\footnote{Ena\v cba~\eqref{brezdim} velja tudi \v ce je
$\rho_0$ karkoli drugega.}.

Ta hamiltonian je \v cisto splo\v sen za gibanje treh (ne nujno nebesnih) to\v ckastih teles pod vplivom sile
gravitacije. Da bodo zvezde res zvezde, si bomo za zgled vzeli res te\v zak planet, Jupiter, ki je \v se
vedno tiso\v c-krat la\v zji od na\v sega Sonca. Kot smiselna se zato zdi omejitev $M_{1,2} \geq 1000$.
Prav tako je smiselno, da so razdalje med zvezdama na za\v cetku res 1.

\subsection{Pribli\v zek zanemarljive mase planeta}

Hamiltonian~\eqref{brezdim} do sedaj eksaktno ohranja energijo. Vendar vidimo, da bo \v clen v potencialu, ki je
sorazmeren z $M_1 M_2$ bistveno ve\v cji od \v clenov, ki so sorazmerni zgolj z $M_{1,2}$ (mogo\v ce za faktor 1000),
saj sta $M_1$ in $M_2$ vsaj reda velikosti $\mathcal{O}(10^3)$. To pomeni, da planet ne bo bistveno vplival na
spremembo gibalne koli\v cine zvezde,
\begin{equation}
    \d\p_{1,2} = \dot{\p}_{1,2}\ \d t = \{\p_{1,2}, V\}\ \d t \approx
        \Big\{\p_{1,2}, -\sfrac{M_1M_2}{|\r_1 - \r_2|}\Big\}\ \d t
\end{equation}
Prav tako se zvezdi glede na planet premikata zelo po\v casi -- efekt planeta bi bil viden \v sele po zelo dolgem
\v casu (po mnogih planetarnih obhodih), saj je sprememba $\r_{1,2}$ ute\v zena z $1/M_{1,2}$. Sprememba koordinat
je
\begin{equation}
    \d \r_{1,2} = \dot{\r}_{1,2}\ \d t = \{\r_{1,2}, T\}\ \d t = \big\{\r_{1,2}, \sfrac{1}{2M_{1,2}}|\p_{1,2}|^2\big\}\ \d t.
\end{equation}
Ker smo pri spremembi gibalne koli\v cine naredili napako, ki je bila reda $\mathcal{O}(10^3)$, je ta napaka sedaj
samo $\mathcal{O}(10^{-6})$, torej smo vpliv planeta upravi\v ceno zanemarili.
\subsubsection{Orbiti zvezd}
Orbiti zvezd, $\r_1(t)$ in $\r_2(t)$, v pribli\v zku zanemarljive mase planeta, lahko izra\v cunamo analiti\v cno brez
uporabe numeri\v cnih orodij. Iz Poissonovih oklepajev dobimo slede\v ce gibalne ena\v cbe:
\begin{align}
    \dot{\r}_{1,2}(t) &= \frac{\p_{1,2}(t)}{M_{1,2}} \\
    \dot{\p}_{1,2}(t) &= \nabla_{1,2} \frac{M_1 M_2}{|\r_1 - \r_2|} = \pm \frac{M_1 M_2}{|\r_1 - \r_2|^3}(\r_2 - \r_1)
    \label{predgibalni}
\end{align}
Da bomo to ena\v cbo lahko re\v sili, uvedemo spremenljivki
\begin{align*}
    \r_c &= M_1\r_1 + M_2\r_2, \\
    \r_* &= M_1\r_1 - M_2\r_2,
\end{align*}
kjer je $\r_c$ sorazmeren s te\v zi\v s\v cem dvozvezdnega sistema, $\r_T = \r_c/(M_1 + M_2)$.Vektorja $\r_1$ in $\r_2$
lahko zapi\v semo kot linearni kombinaciji $\r_c$ in $\r_*$,
\begin{align*}
    \r_1 &= \sfrac{1}{2M_1}(\r_c + \r_*) \\
    \r_2 &= \sfrac{1}{2M_2}(\r_c - \r_*).
\end{align*}
Vektorjema $\r_c$ in $\r_*$ ustrezata gibalni koli\v cini $\p_c$ in $\p_*$. Dobimo ju iz gibalnih ena\v cb za $\r_c$
in $\r_*$:
\begin{align}
    \dot{\r}_c &= M_1\dot{\r}_1 + M_2\dot{\r}_2 = M_1\sfrac{\p_1}{M_1} + M_2\sfrac{\p_1}{M_2} = \p_1 + \p_2 = \p_c\ , \\
    \dot{\r}_* &= M_1\dot{\r}_1 - M_2\dot{\r}_2 = M_1\sfrac{\p_1}{M_1} - M_2\sfrac{\p_1}{M_2} = \p_1 - \p_2 = \p_*\ ,
\end{align}
kjer vidimo, da mora biti $\p_c$ konstanta gibanja, \v ce je izpeljava pravilna. Impulza $\p_1$ in $\p_2$ lahko
izrazimo kot
\begin{align}
    \p_1 &= \sfrac{1}{2}(\p_c + \p_*), \notag \\
    \p_2 &= \sfrac{1}{2}(\p_c - \p_*).
\end{align}
To lahko vstavimo v ena\v cbi~\eqref{predgibalni}, vendar moramo za to vedeti \v se kako se zapi\v se $\r_1 - \r_2$
z $\r_c$ in $\r_*$. V te\v zi\v s\v cnem sistemu $\r_c = 0$ velja
\[
    \r_1 - \r_2 = \sfrac{1}{2}\big(\sfrac{1}{M_1} + \sfrac{1}{M_2}\big)\r_* = \r_*/2\mu.
\]
Gibalne ena\v cbe za impulza zvezd lahko sedaj prepi\v semo v
\begin{align*}
    \dot{\p}_1 &= (\dot{\p}_c + \dot{\p}_*)/2 = \frac{M_1M_2}{|\r_*/2\mu|^3}(-\r_*/2\mu)
        = -\frac{4\mu^2 M_1M_2}{|\r_*|^2}\ \hat{\r}_*\ , \\
    \dot{\p}_2 &= (\dot{\p}_c - \dot{\p}_*)/2 = \frac{M_1M_2}{|\r_*/2\mu|^3}\r_*/2\mu
        = \frac{4\mu^2 M_1M_2}{|\r_*|^2}\ \hat{\r}_*\ ,
\end{align*}
kjer je $\hat{\r}_* \equiv \r_*/|\r_*|$ enotski vektor v smeri $\r_*$. \v Ce ti dve ena\v cbi se\v stejemo, nemudoma
dobimo $\dot{\p}_c = 0$, kar smo tudi pri\v cakovali. Nadaljujemo z razliko prej\v snjih ena\v cb, po katerih
dobimo $\dot{\p}_*$,
\begin{equation*}
    \dot{\p}_* = - 8\mu^2\frac{M_1M_2}{|\r_*|^2}\ \hat{\r}_*.
\end{equation*}
Brez izgube splo\v snosti lahko izhodi\v s\v ce sistema postavimo v te\v zi\v s\v ce ob \v casu $t = 0$, tj.
$\r_c (t = 0) = 0$. Lahko si izberemo tudi pravi te\v zi\v s\v cni sistem, tj. $\dot{\r}_c = 0$, kar pomeni
tudi $\p_c = 0$. Za opis gibanja $\r_1$ in $\r_2$ sedaj zado\v s\v ca le $\r_*$, katerega Newtonov zakon je
\begin{equation}
    \ddot{\r}_* = -8\mu^2 \frac{M_1M_2}{|\r_*|^2}\hat{\r}_*\ = - \frac{\alpha}{|\r_*|^2}\hat{\r}_*\ ,
    \label{newton}
\end{equation}
Ta problem je dvo-dimenzionalen (ker je potencial centralen), zaradi simetrije problema se nam ponujajo polarne
koordinate $\r_*(t) = \r_* (\rho(t), \varphi(t))$. Pri\v celi bomo s prehodom v polarne koordinate:
\begin{equation}
    \ee_\rho = \cos\varphi\ \ee_x + \sin\varphi\ \ee_y, \quad \ee_\varphi = -\rho\sin\varphi\ \ee_x + \rho\cos\varphi\ \ee_y.
\end{equation}
Potem lahko na\v s krajevni vektor prepi\v semo v
\begin{align}
    \r_* &= \rho \cos\varphi\ \ee_x + \rho \sin\varphi\ \ee_y \notag \\
         &= \rho\ \ee_\rho, \\
    \dot{\r}_* &= \dot{\rho}\ee_\rho + \dot{\varphi}\ee_\varphi, \\
    \ddot{\r}_* &= (\ddot{\rho} - \rho\dot{\varphi}^2)\ee_\rho + \bigg(\ddot{\varphi} + 
        2\frac{\dot{\rho}\dot{\varphi}}{\rho}\bigg)\ee_\varphi.
    \label{polarne}
\end{align}
Baza v tem primeru ni normirana ($|\ee_\varphi| \neq 1$), vendar $|\r_*| = \rho$. Zaradi tega dobimo
in $\hat{\r}_* = \r_*/\rho = \ee_\rho$. S procesom "`reverse engineering"'
ena\v cbe~\eqref{newton} lahko zapi\v semo Lagrangeian
\begin{equation*}
    L = \frac{1}{2}|\dot{\r}_*|^2 + \frac{\alpha}{\rho} = \frac{1}{2}(\dot{\rho}^2 + \dot{\varphi}^2\rho^2) +
        \frac{\alpha}{\rho}.
\end{equation*}
Tu $\varphi$ ne nastopa eksplicitno, kar pomeni $\dot{p}_\varphi = 0$ oz. ohranitev vrtilne koli\v cine:
\begin{equation}
    \frac{\d}{\d t}\underbrace{\bigg(\der{L}{\dot{\varphi}}\bigg)}_{= p_\varphi} -
        \underbrace{\der{L}{\varphi}}_{= 0} = 0, \quad p_\varphi = \dot{\varphi}\rho^2,
    \label{vrtilna}
\end{equation}
kar pomeni $\dot{\varphi} = p_\varphi/\rho^2$. Opomba: $\p_* = p_\rho\ee_\rho + p_\varphi\ee_\varphi$. Gibalni zakoni
za $\rho$ so
\begin{align}
    \frac{\d}{\d t}\bigg(\der{L}{\dot{\rho}}\bigg) - \der{L}{\rho} &= 0, \notag \\
    \ddot{\rho} - \dot{\varphi}^2\rho - \bigg(-\frac{\alpha}{\rho^2}\bigg) &= 0, \notag \\
    \ddot{\rho} - \dot{\varphi}^2\rho + \frac{\alpha}{\rho^2} &= 0.
    \label{gibalna}
\end{align}
Ena\v cbo~\eqref{vrtilna} uporabimo za da se znebimo $\varphi$,
\begin{align}
    \frac{\d}{\d t} &= \dot{\varphi}\der{}{\varphi} = \frac{p_\varphi}{\rho^2}\der{}{\varphi}, \\
    \frac{\d^2}{\d t^2} &= \frac{p^2_\varphi}{\rho^4}\bigg(\der[2]{}{\varphi} - \frac{2}{\rho}\der{\rho}{\varphi}
        \der{}{\varphi}\bigg).
\end{align}
Zmenili se bomo $\partial \bullet/\partial \varphi = \bullet'$, da bo manj pisanja. Ko to vstavimo v
ena\v cbo~\eqref{gibalna} dobimo
\begin{equation}
    \frac{p^2_\varphi}{\rho^4}\bigg(\rho'' - \frac{2}{\rho}(\rho')^2 - \rho\bigg) + \frac{\alpha}{\rho^2} = 0.
    \label{skoraj}
\end{equation}
Uvedemo novo spremenljivko $\rho = 1/u$,
\begin{align}
    \rho' &= -\frac{1}{u^2}u' \notag \\
    \rho'' &= 2\frac{1}{u^3}(u')^2 - \frac{1}{u^2}u''.
\end{align}
Sedaj to vstavimo v ena\v cbo~\eqref{skoraj}, kar nam da izraz
\begin{equation*}
    p^2_\varphi u^4\bigg[2\frac{u'^2}{u^3} - \frac{u''}{u^2} - 2\frac{u'^2}{u^3} - \frac{1}{u}\bigg] + \alpha u^2 = 0,
\end{equation*}
ki pa se pokraj\v sa v
\begin{equation}
    u'' + u - \frac{\alpha}{p^2_\varphi} = 0.
\end{equation}
Re\v sitev te ena\v cbe je seveda elipsa,
\begin{equation}
    \rho(\varphi(t)) = \frac{p_\varphi^2/\alpha}{1 + \varepsilon\cos(\varphi(t))},
\end{equation}
Parameter $\varepsilon$ je ekscentri\v cnost orbite, ki so zaklju\v cene (vezano gibanje) samo za
$0 \leq \varepsilon < 1$. Seveda se bomo omejili samo na take.
Kot $\varphi(t)$ je re\v sitev ena\v cbe
\begin{equation}
    \dot{\varphi} = \frac{\alpha^2}{p_\varphi^3}\big(1 + \varepsilon\cos\varphi\big)^2.
    \label{fi}
\end{equation}
Ena\v cbo~\eqref{fi} lahko re\v simo z integralom
\begin{equation}
    \int_0^{\varphi}\frac{\d \tilde{\varphi}}{\big(1 + \varepsilon\cos\tilde{\varphi}\big)^2} = \frac{\alpha^2}{p_\varphi^3}t.
\end{equation}
Ta integral je k sre\v ci analiti\v cen in tabeliran v matemati\v cnem priro\v cniku~\cite{bronstejn}. Za
$0 \leq \varepsilon < 1$ je re\v sitev
\begin{align}
    t\frac{\alpha^2}{p^3_\varphi} &= \bigg[\frac{\varepsilon\sin\varphi}{(\varepsilon^2 - 1)(1 + \varepsilon\cos\varphi)}
            +\frac{1}{1 - \varepsilon^2}\frac{2}{\sqrt{1 - \varepsilon^2}}\arctan\bigg(\frac{(1 - \varepsilon)\tan\varphi/2}
            {\sqrt{1 - \varepsilon^2}}\bigg)\bigg] \notag \\
        &= F(\varphi; \varepsilon).
    \label{fi2}
\end{align}
Kon\v cno lahko sedaj zapi\v semo $\varphi = F^{-1}(t\alpha^2/p^3_\varphi; \varepsilon)$. To moramo ra\v cunati
numeri\v cno, imamo pa to smolo, da je ta funkcija zaradi $\tan\varphi/2$ v funkciji $\arctan(x)$ smiselna samo za
$\varphi \in [-\pi,\pi)$, sicer pa mo\v cno oscilira. Pravilna je kadar $t \in [-T/2, T/2)$, kjer je $T$ \v cas enega
obhoda. Tako bomo \v cas pri ra\v cunanju te ena\v cbe z modulom omejili na ta interval. Vendar pa je ta \v casovni
interval grd -- raje bomo vzeli $t \in [0, T)$, tako da bomo rezultatu pri\v steli $\frac{T\alpha^2}{2p^3_\varphi}$.
Zaradi osebne preference bi tudi $\varphi$ radi merili tako, da $\varphi \in [0, 2\pi)$, zato bomo v ena\v cbi~\eqref{fi2}
krivuljo prestavili v desno za fazo $\pi$, torej $\varphi \mapsto \varphi - \pi$. Tako smo se znebili nezveznosti pri
$\pi$. Sedaj lahko izra\v cunamo \v cas obhoda (oz. periodo). Veljati mora torej
\begin{equation}
    t = \frac{p_\varphi^3}{\alpha^2}\bigg[\frac{\varepsilon\sin(\varphi-\pi)}{(\varepsilon^2 - 1)\big(1 +
        \varepsilon\cos(\varphi - \pi)\big)} +\frac{2}{(1 - \varepsilon^2)^{3/2}}\arctan\bigg(\frac{(1 - \varepsilon)
    \tan\frac{\varphi-\pi}{2}}{\sqrt{1 - \varepsilon^2}}\bigg)\bigg]+\frac{T}{2}
    \label{casovna}
\end{equation}
Obhodni \v cas dolo\v cimo tako, da $t(\varphi = 0) = 0$, oz. $t(\varphi = 2\pi) = T$. To pomeni
\[
    \frac{2}{(1 - \varepsilon^2)^{3/2}}\underbrace{\arctan(-\infty)}_{-\pi/2} = -\frac{T\alpha^2}{2p^3_\varphi},
\]
od koder pa z lahkoto poka\v zemo
\begin{equation}
    T = \frac{2\pi p^3_\varphi}{\alpha^2(1 - \varepsilon^2)^{3/2}}.
    \label{leto}
\end{equation}
\subsubsection{Za\v cetni pogoji}
Da bomo res imeli vezana stanja, mora biti ekscentri\v cnost orbite $\varepsilon \in [0, 1)$. Ta je definirana kot
\[
    \varepsilon = \sqrt{1 + \frac{2Ep_\varphi^2}{\alpha^2}},
\]
od koder lahko slepamo, da mora biti
\[
    -\frac{\alpha^2}{2p^2_\varphi} \leq E < 0.
\]
$E$ je energija sistema, ki jo lahko izra\v cunamo iz za\v cetnih pogojev zvezd:
\begin{equation}
    E = \frac{1}{2}\big(p_\rho^2 + p_\varphi^2/\rho^2\big) - \frac{\alpha}{\rho}
\end{equation}
Ker je razdalja med zvezdama na za\v cetku $1$, je $\rho = 2\mu$ in $\r_*(t = 0) = 2\mu\ee_x$,
torej mora veljati
\begin{equation}
    \rho(\varphi = 0) = 2\mu = \frac{p_\varphi^2/\alpha}{1 + \varepsilon\cos(\varphi-\pi)}\bigg|_{\varphi = 0}.
\end{equation}
To lahko prepakiramo v identiteto
\begin{equation}
    p_\varphi = \sqrt{2\mu\alpha(1 - \varepsilon)}.
\end{equation}
s katero lahko povsod eliminiramo $p_\varphi$:
\begin{align}
    \rho(\varphi) &= \frac{2\mu(1 - \varepsilon)}{1 + \varepsilon\cos(\varphi - \pi)} \\
    T &= \frac{2\pi}{\sqrt{M_1 + M_2}}\bigg(\frac{1 - \varepsilon}{1 - \varepsilon^2}\bigg)^{3/2}
	\label{obhodni}
\end{align}

Sedaj smo dvozvezden sistem v celoti re\v sili, prosta parametra sta le $M_1$ in $M_2$, za\v cetni pogoj pa je odvisen
le od ekscentri\v cnosti orbite, $\varepsilon$.

\subsubsection{Ra\v cunanje orbit zvezd}
Na vsakem \v casovnem koraku bomo morali re\v siti ena\v cbo~\eqref{casovna}, ki je k sre\v ci monotona. Paziti moramo,
da ne pademo ven iz tega intervala. Ni\v cle lahko u\v cinkovito ra\v cunamo z analiti\v cnim odvodom -- ponujajo se
nam Newtonova, Halleyeva in Baileyeva metoda. Za slednji potrebujemo \v se drugi odvod. Ker velja $\partial t/\partial
\varphi = (\partial \varphi/\partial t)^{-1}$ lahko zapi\v semo
\begin{align}
    \frac{\alpha^2 t'}{p_\varphi^3} = \frac{\alpha^2}{p_\varphi^3} \der{t}{\varphi} &= \frac{1}{[1 +
        \varepsilon\cos(\varphi - \pi)]^2}
	\label{odvodcasa} \\
    \frac{\alpha^2 t''}{p_\varphi^3} = \frac{\alpha^2}{p_\varphi^3} \der[2]{t}{\varphi} &=
        \frac{2\varepsilon\sin(\varphi - \pi)}{[1 + \varepsilon\cos(\varphi - \pi)]^3}
\end{align}

Na koncu smo uporabili kar Newtonovo metodo, preciznost je $10^{-12}$.

\subsubsection{Dinamika planeta}
Sedaj smo izra\v cunali vse, kar potrebujemo za izra\v cun "`zunanjega"' potenciala v kateraga vr\v zemo na\v s planet.
Hamiltonka se glasi
\begin{equation}
    H_3 = \frac{1}{2}\big(p_\zeta^2 + p_\psi^2/\zeta^2\big) + V_\odot(\r_3,t),
\end{equation}
kjer je potencial $V_\odot = V_\odot(\r, t)$ enak
\begin{equation}
    V_\odot (\r, t) \equiv -\Bigg[\frac{M_1}{\big|\r - \frac{1}{2M_1}\r_*(t)\big|}
        + \frac{M_2}{\big|\r + \frac{1}{2M_2}\r_*(t)\big|}\Bigg].
\end{equation}
Kljub temu, da \v cas eksplicitno nastopa v hamiltonki, se Poissonov oklepaj pri tem ne spremeni. Uvedli smo \v se
polarne koordinate planeta:
\[
    \r_3 = \zeta\cos\psi\ \ee_x + \zeta\sin\psi\ \ee_y = \zeta\ \ee_\zeta.
\]
Gibalne zakone dobimo iz Poissonovih oklepajev $\{\bullet, T_3\}$ in $\{\bullet, V_\odot\}$:
\begin{alignat}{3}
    \{\zeta, T_3\}        &= \der{T_3}{p_\zeta} = p_\zeta, \qquad&
    \{\zeta, V_\odot\}    &= 0, \qquad \notag \\
    \{\psi, T_3\}         &= \der{T_3}{p_\psi} = p_\psi/\zeta^2, \qquad&
    \{\psi, V_\odot\}     &= 0, \notag \\
	\{p_\zeta, T_3\} &= -\der{T_3}{\zeta} = p^2_\psi/\zeta^3, \qquad&
    \{p_\zeta, V_\odot\}  &= -\der{V_\odot}{\zeta}, \notag \\
	\{p_\psi, T_3\}       &= 0, \qquad&
    \{p_\psi, V_\odot\}   &= -\der{V_\odot}{\psi},
    \label{oklepaji}
\end{alignat}
Potencial $V_\odot$ se v polarnih koordinatah zapi\v se kot
\begin{align}
    V_\odot     &= -\frac{M_1}{|\r_{31}|} -\frac{M_2}{|\r_{32}|}, \\
    |\r_{31}|^2 &= \zeta^2 + \frac{1}{4M_1^2}\rho^2 - \frac{1}{M_1}\rho\zeta\cos(\psi - \varphi),
	\label{r31} \\
    |\r_{32}|^2 &= \zeta^2 + \frac{1}{4M_2^2}\rho^2 + \frac{1}{M_2}\rho\zeta\cos(\psi - \varphi).
	\label{r32}
\end{align}
Od tod sledi
\begin{align}
    -\der{V_\odot}{\zeta} &= (-1)\cdot\bigg[+\frac{1}{2}\frac{M_1}{|\r_{31}|^3}\bigg(2\zeta -
        \frac{\rho}{M_1}\cos(\varphi - \psi)\bigg) + \frac{1}{2}\frac{M_2}{|\r_{32}|^3}
        \bigg(2\zeta + \frac{\rho}{M_2}\cos(\varphi - \psi)\bigg)\bigg] \notag \\
    &= \frac{\rho}{2}\bigg(\frac{1}{|\r_{31}|^3} - \frac{1}{|\r_{32}|^3}\bigg)\cos(\varphi - \psi) -
        \zeta\bigg(\frac{M_1}{|\r_{31}|^3} + \frac{M_2}{|\r_{32}|^3}\bigg) \\
        -\der{V_\odot}{\psi} &= (-1)\cdot\bigg[-\frac{1}{2}\frac{M_1}{|\r_{31}|^3}\frac{\rho\zeta}{M_1}\sin(\varphi - \psi)
        + \frac{1}{2}\frac{M_2}{|\r_{32}|^3}\frac{\rho\zeta}{M_2}\sin(\varphi - \psi) \bigg] \notag \\
    &= \frac{\rho\zeta}{2}\bigg(\frac{1}{|\r_{31}|^3} - \frac{1}{|\r_{32}|^3}\bigg)\sin(\varphi - \psi)
\end{align}

\subsection{Pribli\v zek povpre\v cnega polja}

Za $M_1 = 2000$ in $M_2 = 1000$ dobimo $p_\varphi \sim 10^4$. Ocenimo maksimalno dovoljeno energijo
planeta kot
\begin{equation}
	E_3 = \frac{1}{2}(p^2_\zeta + p_\psi^2/\zeta^2) - \mu/\zeta
\end{equation}
Energija mora biti negativna za vezan sistem, tj. $T_3 < V_\odot$. Za za\v cetne pogoje si brez izgube
splo\v snosti izberemo $p_\zeta = 0$, od koder dobimo oceno za $p_\psi(t = 0)$.
\[
	p_\psi^2 < 2\zeta^2 \mu/\zeta = 2\mu\zeta
\]
oziroma malce druga\v ce
\begin{equation}
	-\sqrt{2\mu\zeta} \lesssim p_\psi \lesssim \sqrt{2\mu\zeta}.
\end{equation}
Oddaljenost planeta od dvozvezdja ne sme biti prevelika, tj. $\zeta \sim 10^2$, reducirana masa zvezd pa je
reda $\mu \sim 10^3$. Odtod dobimo oceno za maksimalno dovoljeno vrtilno koli\v cino $p_\psi^\text{max}
\sim 10^{5/2} \approx 300$. Vidimo, da je $p_\psi^\text{max} \sim 10^2 \ll p_\varphi \sim 10^4$. To pomeni,
da bo planet gotovo spreminjajo\v ce se polje \v cutil kot \v casovno povpre\v cno polje, torej lahko
\v casovno odvisnost izintegriramo. Ozna\v cimo $\langle \bullet \rangle_t$ za operator \v casovnega povpre\v cja.
\v Casovno povpre\v cno polje bomo definirali kot
\begin{equation}
	\langle V_\odot(\r)\rangle_t \equiv \frac{1}{T}\int_0^T \d t\ V_\odot (\r,t) = \frac{1}{T}\int_0^{2\pi}\d \varphi\
		\bigg|\der{\varphi}{t}\bigg|^{-1} V_\odot(\r, \varphi(t)),
	\label{polje}
\end{equation}
saj je v eni periodi vsa informacija, ki jo potrebujemo. S pomo\v cjo ena\v cb~\eqref{obhodni} in~\eqref{odvodcasa}
lahko ena\v cbo~\eqref{polje} prepi\v semo v
\begin{equation}
	\langle V_\odot(\r)\rangle_t = (1 - \varepsilon)^{3/2} \frac{1}{2\pi}\int_0^{2\pi} \d \varphi\
		\frac{V_\odot(\r,\varphi)}{[1 + \varepsilon\cos(\varphi - \pi)]^2}.
\end{equation}
Tekom tega poro\v cila smo izra\v cunali vse, kar potrebujemo, da ta integral eksplicitno zapi\v semo, vendar
pa je grd. Tudi \v ce analiti\v cna re\v sitev obstaja, je neprakti\v cna za uporabo (program {\tt Mathematica},
ga ni uspel izra\v cunati po dveh celih urah), zato bi ga morali ra\v cunali kar numeri\v cno. Tudi v Poissonovih
oklepajih ga lahko \v se vedno uporabimo:
\begin{equation}
	\big\{A,\langle V_\odot\rangle_t\big\} = \frac{(1 - \varepsilon)^{3/2}}{2\pi}\int_0^{2\pi}\d \varphi\
		\frac{1}{[1 + \varepsilon\cos(\varphi - \pi)]^2}\{A, V_\odot(\r,\varphi)\},
\end{equation}
kjer je $A$ neka kanoni\v cna spremenljivka. 

\subsubsection{Kro\v zne orbite zvezd}
Kljub temu pa obstaja relativno lepa analiti\v cna re\v sitev za kro\v zne orbite zvezdnega sistema ($\varepsilon = 0$).
\v Ce je planet dale\v c pro\v c od zvezd, ekscentri\v cnost zvezdnih orbit nima nobenega vpliva, tako da je ta
pribli\v zek v na\v sem primeru upravi\v cen, za $\zeta \gtrsim 10$. Takrat vzamemo limito $\varepsilon \to 0$ za na\v se
koli\v cine:
\begin{align}
	\lim_{\varepsilon \to 0} \rho (\varphi) &= 2\mu = \text{konst.} \\
	\lim_{\varepsilon \to 0} \langle V_\odot \rangle_t &= \frac{1}{2\pi}\int_0^{2\pi} \d \varphi\ \lim_{\varepsilon \to 0}
		V_\odot
	\label{povprecni:potencial}
\end{align}
Za izra\v cun $\lim_{\varepsilon \to 0} V_\odot$ potrebujemo limiti $|\r_{31}|^2$ in $|\r_{32}|^2$ iz ena\v cb~\eqref{r31}
in~\eqref{r32}:
\begin{align}
	\lim_{\varepsilon \to 0} |\r_{31}|^2 &= \zeta^2 + \frac{\mu^2}{M_1^2} - 2\frac{\mu}{M_1}\zeta
			\cos(\varphi - \psi) \notag \\
		&= \Big(\zeta - \frac{\mu}{M_1}\Big)^2 + 4\frac{\mu}{M_1}\zeta\sin^2\frac{\varphi - \psi}{2}, \\
	\lim_{\varepsilon \to 0} |\r_{32}|^2 &= \zeta^2 + \frac{\mu^2}{M_2^2} + 2\frac{\mu}{M_2}\zeta
			\cos(\varphi - \psi) \notag \\
		&= \Big(\zeta + \frac{\mu}{M_2}\Big)^2 - 4\frac{\mu}{M_2}\zeta\sin^2\frac{\varphi - \psi}{2}.
	\label{r32:2mu}
\end{align}
Integral~\eqref{povprecni:potencial} lahko sedaj s pomo\v cjo integrala ki je spet tabeliran v
priro\v cniku~\cite{bronstejn}
\begin{align}
	\frac{1}{a_k}\int_0^\theta \frac{\d x}{\sqrt{1 \mp b_k\sin^2 c_kx}} = \frac{1}{a_kc_k} F(c_k \theta; \pm b_k),
	\qquad  |b_k| < 1
	\label{elipticni}
\end{align}
kjer je $F(x;k)$ nepopolni elipti\v cni integral prve vrste. \v Ce privzamemo
\begin{alignat}{3}
	x &= \varphi - \psi, &\quad c_{1,2} &= 1/2, \notag \\
	a_1 &= 2\pi(\zeta - \mu/M_1), &\quad  a_2 &= 2\pi(\zeta + \mu/M_2), \notag \\
	b_1 &= 4\frac{\mu\zeta/M_1}{(\zeta - \mu/M_1)^2}, &\quad  b_2 &= 4\frac{\mu\zeta/M_2}{(\zeta + \mu/M_2)^2},
	\label{konstante}
\end{alignat}
potem lahko integral~\eqref{povprecni:potencial} s pomo\v cjo ena\v cbe~\eqref{elipticni} re\v simo
\begin{align}
	\lim_{\varepsilon \to 0} \langle V_\odot \rangle_t = &- \frac{2M_1}{a_1}\Big[F(\psi/2; -b_1) +
			F(\pi - \psi/2; -b_1)\Big] \notag \\
		&- \frac{2M_2}{a_2}\Big[F(\psi/2; b_2) + F(\pi - \psi/2; b_2)\Big],
	\label{naivno}
\end{align}
pri tem dobimo \v se pogoj $\zeta \neq \pm \mu/M_1$ oz. $\zeta \neq \pm \mu/M_2$, ki pride iz $|b_k| < 1$ in
iz tega, da je deljenje z ni\v c tabu. Ta pogoj je avtomati\v cno izpolnjen za $\zeta \gg 1$, saj je
$\mu/M_{1,2} \sim 1$.

\paragraph{Gibalne ena\v cbe:}
Sam elipti\v cni integral je transcendentna funkcija. Za gibalne ena\v cbe potrebujemo \v se odvode
\begin{equation}
	\{p_\psi, \avgt{V_\odot}\} = -\der{}{\psi}\avgt{V_\odot}, \qquad 
	\{p_\zeta, \avgt{V_\odot}\} = -\der{}{\zeta}\avgt{V_\odot}.
\end{equation}
Da re\v simo ti dve ena\v cbi pa potrebujemo odvode elipti\v cnih integralov. Velja
\begin{align}
	\der{F(x;k)}{x} &= \frac{1}{\sqrt{1 - k\sin^2x}}, \\
	\der{F(x;k)}{k} &= \frac{\sin 2x}{4(k - 1)\sqrt{1 - k\sin^2x}} - \frac{1}{2k}\bigg[F(x;k)
		+ \frac{1}{k-1}E(x;k)\bigg], \\
		\der{F(x;-k)}{k} &= \frac{\sin 2x}{4(k+ 1)\sqrt{1 + k\sin^2x}} - \frac{1}{2k}\bigg[F(x;-k)
		- \frac{1}{{k+1}}E(x;-k)\bigg], \\
	\der{a_{1,2}}{\zeta} &= 2\pi, \\
	\der{b_{1,2}}{\zeta} &= \frac{4\frac{\mu}{M_{1,2}}(\zeta \mp \mu/M_{1,2})^2 - 4\zeta\frac{\mu}{M_{1,2}} \cdot
	2(\zeta \mp \mu/M_{1,2})}{(\zeta \mp \mu/M_{1,2})^4} \notag \\
	&= b_{1,2}\bigg(\frac{1}{\zeta} - \frac{2}{\zeta \mp \mu/M_{1,2}}\bigg) \notag \\
	&= b_{1,2}(1/\zeta - 4\pi/a_{1,2}),
\end{align}
kjer je $E(x;k)$ nepopoln elipti\v cni integral druge vrste. Odvode potem lahko zapi\v semo kot
\begin{align}
	\der{}{\psi}\lim_{\varepsilon \to 0}\avgt{V_\odot} = &-\frac{2M_1}{a_1}\bigg[\der{F(\psi/2;-b_1)}{\psi/2}
		\der{\psi/2}{\psi} + \der{F(\pi - \psi/2;-b_1)}{(\pi - \psi/2)}\der{(\pi - \psi/2)}{\psi}\bigg] \notag \\
	&- \frac{2M_2}{a_2}\bigg[\der{F(\psi/2;b_2)}{\psi/2}\der{\psi/2}{\psi} +
		\der{F(\pi - \psi/2;b_2)}{(\pi - \psi/2)}\der{(\pi - \psi/2)}{\psi}\bigg] \notag \\
	= &+ \frac{M_1}{a_1}\der{F(x;-b_1)}{x}\bigg|_{x = \psi/2}^{x = \pi - \psi/2} + \frac{M_2}{a_2}
		\der{F(x;b_2)}{x}\bigg|_{x = \psi/2}^{x = \pi - \psi/2} \notag \\
	= &- \frac{2M_1/a_1}{\sqrt{1 + b_1\sin^2(\psi/2)}} - \frac{2M_2/a_2}{\sqrt{1 - b_2\sin^2(\psi/2)}}, \\
	\der{}{\zeta}\lim_{\varepsilon \to 0} \avgt{V_\odot} = &+\frac{2M_1}{a_1^2}\der{a_1}{\zeta}\bigg[F(\psi/2; -b_1)
		+ F(\pi - \psi/2; -b_1)\bigg] \notag \\
	&+ \frac{2M_2}{a_2^2} \der{a_2}{\zeta}\bigg[F(\psi/2;b_2) + F(\pi - \psi/2;b_2)\bigg] \notag \\
	&- \frac{2M_1}{a_1}\bigg[\der{F(\psi/2;-b_1)}{b_1} +
		\der{F(\pi - \psi/2;-b_1)}{b_1}\bigg]\der{b_1}{\zeta} \notag \\
	&-\frac{2M_2}{a_2}\bigg[\der{F(\psi/2;b_2)}{b_2} +
		\der{F(\pi - \psi/2;b_2)}{b_2}\bigg]\der{b_2}{\zeta}
\end{align}
Vidimo, da kjub pribli\v zku kro\v znih zvezdnih orbit, kon\v cne re\v sitve niso trivialne.

\paragraph{Popravek:}
\v Casovno povpre\v cen potencial~\eqref{naivno} ima eno hibo. Vsa informacija o tem, kje se zvezde
nahajajo (v povpre\v cju) je podana z ni\v clami $a_{1,2}$. Vidimo, da ima $a_1$ ni\v clo za $\zeta > 0$, vendar
pa ima $a_2$ ni\v clo za $\zeta < 0$, kar pa ni prav. Po tistem, ko izintegriramo kot $\varphi$, so elipti\v cni
integrali na celotnem na\v sem definicijskem kon\v cni.

K sre\v ci je $\r_3$ (in posledi\v cno tudi $|\r_{32}|$ in $V_\odot(\r,t)$) invarianten\footnote{Povpre\v ceni
potencial $\avgt{V_\odot}$ pa nima te simetrije, ker jo integriranje zlomi, zaradi \v cesar ga moramo znova
izra\v cunati.} na transformacijo
\begin{align*}
	(\zeta, \psi) \mapsto (-\zeta, \psi - \pi).
\end{align*}
Pri tem vektor $\r$ \v se vedno ka\v ze v isto to\v cko, vendar pa dobimo pravilno divergenco za drugo zvezdo.
To pomeni, da moramo popraviti ena\v cbo~\eqref{r32:2mu}
\begin{equation}
	|\r_{32}|^2 = \Big(\zeta - \frac{\mu}{M_2}\Big)^2 + 4\frac{\mu}{M_2}\sin^2\frac{\varphi - \psi + \pi}{2}
\end{equation}
in temu primerno spremeniti ena\v cbe~\eqref{konstante}:
\begin{alignat}{3}
	x_1 &= \varphi - \psi, \qquad& x_2 &= \varphi - \psi + \pi, \notag \\
	a_1 &= 2\pi(\zeta - \mu/M_1), \qquad& a_2 &= 2\pi(\zeta - \mu/M_2), \notag \\
	b_1 &= 4\frac{\mu\zeta/M_1}{(\zeta - \mu/M_1)^2}, \qquad& b_2 &= 4\frac{\mu\zeta/M_2}{(\zeta - \mu/M_2)^2},
\end{alignat}
$c_{1,2}$ pa je \v se vedno enak $c_{1,2} = c = 1/2$. Od tod dobimo popravljen potencial
\begin{align}
	\lim_{\varepsilon \to 0} \avgt{V_\odot} = &- \frac{2M_1}{a_1}\left[F\left(\frac{\psi}{2}; -b_1\right)
		+ F\left(\frac{2\pi - \psi}{2}; -b_1\right)\right] \notag \\
	&- \frac{2 M_2}{a_2}\left[F\left(\frac{\psi - \pi}{2}; -b_2\right)
		+ F\left(\frac{3\pi - \psi}{2}; -b_2\right)\right],
\end{align}
ki je pravilno definiran na $\zeta > 0$ in $\psi \in [0, 2\pi)$. Definirajmo
\begin{align}
	\Lambda_k (\zeta, \psi) \equiv &\frac{2\pi}{a_k}F(\psi/2;b_k) - \der{b_k}{\zeta}\der{F(\psi/2;-b_k)}{b_k} \notag \\
	= &\frac{1}{b_k + 1}\Bigg[\frac{\sin\psi}{4\sqrt{1 + b_k\sin^2(\psi/2)}}
		\bigg(1 - \frac{b_k}{\zeta}\bigg) + E(\psi/2; -b_k)\bigg(1 - \frac{1}{2\zeta}\bigg)\Bigg] \notag \\
	&+\frac{1}{2b_k}F(\psi/2; -b_k)
\end{align}
Popraviti je treba \v se nekatere odvode:
\begin{align}
	\der{}{\psi}\lim_{\varepsilon \to 0} \avgt{V_\odot} = &- \frac{2M_1/a_1}{\sqrt{1 + b_1\sin^2(\psi/2)}}
		- \frac{2M_2/a_2}{\sqrt{1 + b_2\sin^2\big((\psi - \pi)/2\big)}} \\
	\der{}{\zeta}\lim_{\varepsilon \to 0} \avgt{V_\odot} = &+ \frac{2M_1}{a_1}\Big[\Lambda_1(\zeta, \psi)
	+ \Lambda_1(\zeta, 2\pi - \psi)\Big] \notag \\
	&+ \frac{2M_2}{a_2}\Big[\Lambda_2(\zeta, \psi - \pi) + \Lambda_2(\zeta, 3\pi - \psi)\Big]
\end{align}

Rezultate za $V_\odot(\r,t)$ bi bilo zanimivo primerjati z rezultati v povpre\v cnem polju, $\avgt{V_\odot}(\r)$ ob
$\varepsilon \to 0$. Gibalni zakoni so isti, le $\partial V_\odot/\partial x$ zamenjamo z odvodi povpre\v cnega polja
$\partial \avgt{V_\odot}/\partial x$.

\subsection{Numeri\v cni izra\v cun}
Ker imamo na planet vpliv zunanjega \v casovno odvisnega potenciala energija planeta ne bo konstantna. Vendar pa
tudi za tretma \v casovno odvisnih hamiltonk uporabimo simplekti\v cne integratorje, saj pojem simplekti\v cnosti
ni omejen na ohranitev energije\footnote{V tem primeru se zelo natan\v cno ra\v cuna $\partial H/\partial t$.}, ampak
na ohranjanje hamiltonske forme~\cite{sirca} (tj. Poissonovega oklepaja), ki se ne spremeni tudi \v ce \v cas
eksplicitno nastopa v Hamiltonianu. Prednost simplekti\v cnih integratorjev je ta, da nam dajo stabilnej\v se
orbite, poleg tega pa jih je enostavneje naprogramirati

Za konstrukcijo simplekti\v cnega integratorja potrebujemo $H(t) = A(t) + B(t)$, kjer $\{A(t), B(t)\} \neq 0$.
Vektor integracijskih spremenljivk je $\mathbf{x} = (\zeta, \psi, p_\zeta, p_\psi)^T$. Za integracijo bomo
uporabili simetrizirane sheme drugega reda\footnote{Napaka za red $m$ je reda $\mathcal{O}(\delta t^{m+1})$.},
$S_2$, \v cetrtega reda, $S_4$ in osmega reda, $S_8$:
\[
    S_2(c,\delta t) \equiv \exp(c\sfrac{\delta t}{2}\{\bullet, A\})\exp(c\delta t\{\bullet, B\})
    \exp(c\sfrac{\delta t}{2}\{\bullet, A\}),
\]
shema \v cetrtega reda $S_4$ je
\begin{equation}
    S_4(\delta t) = S_2(x_0, \delta t) S_2 (x_1, \delta t) S_2 (x_0, \delta t), \quad
        x_0 = \frac{1}{2 - 2^{1/3}},\ x_1 = -\frac{2^{1/3}}{2 - 2^{1/3}},
\end{equation}
za shemo osmega reda, $S_8$, pa bomo vzeli koeficiente $w_1$, re\v sitev `$\mathrm{A}$' iz \v clanka~\cite{yoshida}
s \v cimer aproksimiramo Liouvillov propagator. Integracija zatorej izgleda tako:
\[
	\mathbf{x}(t + \delta t) = S_m(\delta t) \mathbf{x}(t) + \mathcal{O}(\delta t^{m+1}).
\]
Opomba:
\[
    \exp(c\delta t\{\bullet, A\})x = x + c\delta t\{x, A\}.
\]
Izbrati si moramo taka $A(t)$ in $B(t)$, ki delujeta na disjunktna hilbertova prostora. Ker smo v polarnih koordinatah,
zato ne bomo mogli vzeti kar $T_3$ in $V_\odot$, ampak bomo morali vzeti efektivni potencial $\Vef$ in radialno
kineti\v cno energijo $T_\zeta$:
\begin{equation}
    T_\zeta = \frac{1}{2}p_\zeta^2, \quad \Vef = \frac{1}{2}\frac{p_\psi^2}{\zeta^2} + V_\odot.
\end{equation}
Potrebujemo \v se vrednosti vse mo\v znih Poissonovih oklepajev. Na sre\v co si lahko pomagamo s tistimi, ki
smo jih nara\v cunali v ena\v cbah~\eqref{oklepaji}:
\begin{alignat}{3}
    \{\zeta, T_\zeta\}        &= p_\zeta, \qquad &
    \{\zeta, \Vef\}           &= 0, \notag \\
    \{\psi, T_\zeta\}         &= 0, \qquad &
    \{\psi, \Vef\}            &= p_\psi/\zeta^2, \notag \\
    \{p_\zeta, T_\zeta\}      &= 0, \qquad&
    \{p_\zeta, \Vef\}         &= p^2_\psi/\zeta^3 - \der{V_\odot}{\zeta}, \notag \\
    \{p_\psi, T_\zeta\}       &= 0, \qquad &
    \{p_\psi, \Vef\}          &= -\der{V_\odot}{\psi}.
\end{alignat}
Vidimo, da je tak razcep hamiltonke $H_3$ res ustrezen. Sedaj smo lahko gotovi, da bo "`drift"' orbite posledica
$\partial H_3 / \partial t \neq 0$ in ne posledica slabega integratorja.

Vseeno pa obstaja mo\v znost, da se bo po dolgem \v casu planet mo\v cno pribli\v zal eni izmed
zvezd, zaradi \v cesar bo planet vrglo oson\v cja po ti. "`slingshot"' efektu. Da se to ne zgodi prekmalu, moramo
imeti \v se dodatno zagotovilo stabilnosti orbit -- adaptiven korak. \v Ce smo preblizu zvezde, ima najmanj\v sa
napaka v kotu/radiju daljnose\v zne posledice na evolucijo. Pri izbiri adaptivnega koraka koramo biti pozorni,
da med drobljenjem koraka na manj\v se ne smemo zadeti ob frekvenco, ki je lastna dvozvezdju~\cite{richardson}.
Zato bomo na za\v cetku vzeli \v casovni korak, ki je v iracionalnem razmerju z obhodnim \v casom zvezde, korake
pa bomo drobili po Rombergovem zaporedju (potence \v stevila $2$). Za primerjavo sem za integrator preizkusil
\v se Runge-Kutta reda 4, ki pa se ni obnesel najbolje.

Orbite planeta niso ve\v c preproste elipse, zato Laplance-Runge-Lenzov vektor ni ve\v c invarianta in ga bomo
tudi spremljali. Z na\v simi opazljivkami $\zeta$, $\psi$, $p_\zeta$ in $p_\psi$ ga v kartezi\v cnih koordinatah
na se\v cni ploskvi $\psi = 0$ zapi\v semo kot
\begin{equation}
	\ee = \left\{\left[p_\zeta^2 + \Big(\frac{p_\psi}{\zeta}\Big)^2\right]\zeta - \zeta p_\zeta^2 - 1\right\}
		\ee_x - p_\zeta p_\psi \ee_y.
\end{equation}
To je v resnici vektor ekscentri\v cnosti, vendar pa se od Laplance-Runge-Lenz-ovega vektorja razlikuje le
za multiplikativno konstanto. Glavna motivacija za ra\v cunanje tega Laplace-Runge-Lenz-ovega vektorja nam
je bila izra\v cun smeri in ekscentri\v cnosti orbite, zato je ra\v cunanje vektorja ekscentri\v cnoti
ustreznenj\v se.

\section{Rezultati}

Nalogo sem re\v seval s programskim jezikom {\tt C}, da bi dosegel \v cim vi\v sjo preciznost za \v cim
kraj\v si \v cas ra\v cunanja sem preizku\v sal prej omenjene integratorje. Kot sem \v ze povedal prej,
se Runge-Kutta reda 4 vede precej slabo, zato sem se posvetil le simplekti\v cnim integratorjem. Drugi
red ni bil zadosten, orbita planeta ni bila dovolj stabilna. Integratorski shemi $S_8$ in $S_4$ sta bili
dokaj primerljivi, vendar je v bli\v zini zvezde shema $S_8$ dajala bolj gladke tirnice, zato sem pristal
kar na $S_8$, ki je najnatan\v cnej\v sa, vendar tudi najpo\v casnej\v sa.

\begin{figure}[H]\centering
	\includegraphics{graf160}
	\caption{}
\end{figure}

\begin{figure}[H]\centering
	\includegraphics{graf180}
	\caption{}
\end{figure}

\begin{figure}[H]\centering
	\includegraphics{graf200}
	\caption{}
\end{figure}

\begin{figure}[H]\centering
	\includegraphics{graf220}
	\caption{}
\end{figure}

\begin{thebibliography}{9}
	\bibitem{bronstejn}
		J. N. Bron\v stejn in K. A. Semendjejev,
		{\em Matemati\v cni priro\v cnik},
		Tehni\v ska zalo\v zba Slovenije v Ljubljani,
		deseti ponatis,
		(1988)

    \bibitem{sirca}
        S. \v Sirca in M. Horvat,
        {\em Ra\v cunske metode za fizike},
        DMFA Zalo\v zni\v stvo,
        (2010)

    \bibitem{yoshida}
        H. Yoshida,
        \emph{Construction of higher order symplectic integrators},
        Phys. Lett. A,
        Vol. 150, no. 5,6,7, str. 262,
        (1990)

    \bibitem{richardson}
        A. S. Richardson and J. M. Finn,
        \emph{Symplectic integrators with adaptive time steps},
        {\tt arXiv:1108.0322v1 [physics.comp-ph]},
        (2011)
\end{thebibliography}

\end{document}
