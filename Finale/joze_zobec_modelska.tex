\documentclass[12pt, a4paper]{article}
\usepackage[slovene]{babel}
\usepackage[utf8]{inputenc}
\usepackage[T1]{fontenc}
\usepackage{amsmath, amssymb}

\renewcommand{\r}{
    \ensuremath{\mathbf{r}}
}

\renewcommand{\L}{
    \ensuremath{\mathcal{L}}
}

\newcommand{\e}{
    \ensuremath{\mathrm{e}}
}

\newcommand{\sfrac}[2]{
    \ensuremath{\textstyle{\frac{#1}{#2}}}
}

\begin{document}

\section{Navodilo}

Razi\v s\v ci prostor kro\v znih orbit planeta okrog dvojne zvezde v pribli\v zku zanemarljive mase
planeta. Koliko parametrov ima model?

\section{Razmislek}

V tem primeru imamo veliko sre\v co, saj je sistem zaprt, kar pomeni $\dot{E} = 0$. Zaradi tega bomo
za re\v sevanje problema napravili simplekti\v cen integrator s pomo\v cjo Trotter-Suzukijevega
razcepa, za katerega pa ne potrebujemo Newtonovega zakona, ampak samo klasi\v cno hamiltonko:
\begin{equation}
    H = \frac{1}{2}\sum_{k = 1}^N \frac{p_k^2}{m_k} - \frac{\kappa}{4\pi}\sum_{k,\ell}
        \frac{m_k m_\ell}{|\r_k - \r_\ell|}.
    \label{hamilton}
\end{equation}
Dinamiko dolo\v cimo prek Liouvillovega operatorja $\L$:
\begin{equation}
    -i\L \equiv \{\bullet, H\}, \quad \dot{x}(t) = -i\L x(t),
    \label{liouville}
\end{equation}
kjer je $\{\bullet, \bullet\}$ Poissonov oklepaj, $x$ pa neka koli\v cina. Iz Ena\v cbe~\eqref{liouville}
dobimo klasi\v cni propagator $U(t)$:
\begin{equation}
    x(t + \delta t) = U(\delta t)\ x(t) = \exp(\delta t\L)\ x(t),
\end{equation}
katerega lahko razstavimo na kineti\v cni in potencialni del. Propagator se potem glasi
\begin{equation}
    \exp(\delta t \L) = \exp(\delta t \{\bullet, T\} + \delta t \{\bullet, V\}),
    \label{propagator}
\end{equation}
zanj pa velja Trotterjeva formula
\begin{equation}
    \exp(\delta t\{\bullet,T\} + \delta t\{\bullet,V\}) = \lim_{n \to \infty}
        \Big(\exp(\sfrac{\delta t}{n}\{\bullet,T\})\exp(\sfrac{\delta t}{n}\{\bullet,V\})\Big)^n.
\end{equation}
Tega izraza seveda ne bomo ra\v cunali, saj bi potrebovali neskon\v cen produk. Raje bi konsistentno z
natan\v cnostjo pobrali le kon\v cno mnogo faktorjev. Tu nam na pomo\v c prisko\v cijo
Baker-Campbell-Hausorff-ova identiteta:
\[
    C = \log{\e^{A + B}} = A + B + \sfrac{1}{2}[A,B] + \sfrac{1}{12}\big[A,[A,B]\big] -
        \sfrac{1}{12}\big[B,[A,B]\big] +\ldots
\]
S pomo\v cjo le-te si lahko zamislimo pogoja
\begin{align}
    \e^C &= \e^{c_1 A}\e^{d_1 B}\e^{c_2 A}\e^{d_2 B}\e^{c_3 A}\cdot \ldots \cdot \e^{c_n A}\e^{d_n B} \\
    \e^C &= \e^{w_1 A}\e^{z_1 B}\e^{w_2 A}\e^{z_2 B}\e^{w_3 A}\cdot \ldots \cdot \e^{z_n B}\e^{w_{n+1} A}
\end{align}
in izra\v cunamo koeficiente $c_i$ in $d_j$ nesimetri\v cne oz. koeficiente $w_i$, $z_i$ simetrizirane
formule da bo to res do reda natan\v cnosti $n$, ki se sorazmeren s \v stevilom prisotnih faktorjev.
Simetrizirane formule so natan\v cnej\v se zato bomo uporabili sledenje. Koeficienti so v splo\v snem
lahko kompleksni. S pomo\v cjo slede\v sih identitet lahko propagator v ena\v cbi~\eqref{propagator}
zapi\v semo kot npr.
\[
    \exp(\delta t\L) \approx \exp\big(\sfrac{\delta t}{2}\{\bullet, T\}\big)
    \exp\big(\delta t\{\bullet, V\}\big)
    \exp\big(\sfrac{\delta t}{2}\{\bullet, T\}\big) + \mathcal{O}(\delta t^3).
\]
Ta shema je Trotter-Suzukijev razcep reda $2$ in jo ozna\v cimo z $S_2(\delta t)$. Na preprost na\v cin
lahko od tod dobimo Trotter-Suzukijev razcep reda $4$,
\[
    S_4 (\delta t) = S_2(x_0 \delta t) S_2(x_1 \delta t) S_2(x_0 \delta t), \quad
    x_0 = \frac{1}{2 - \sqrt[3]{2}},\ x_1 = -\frac{\sqrt[3]{2}}{2 - \sqrt[3]{2}},
\]
ki ga bomo uporabljali tekom te doma\v ce naloge, saj je \v ze kar precej natan\v cen.

\subsection{Prosti parametri}

Da preverimo\v stevilo prostih parametrov moramo hamiltonko~\eqref{hamilton} pretvoriti v brezdimenzijsko
obliko.

\end{document}
