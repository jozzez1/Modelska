\documentclass[a4 paper, 12pt]{article}
\usepackage[slovene]{babel}
\usepackage[utf8]{inputenc}
\usepackage[T1]{fontenc}
\usepackage[small]{caption}
\usepackage[pdftex]{graphicx}
\usepackage{amssymb, fullpage, float, pdflscape, subcaption, amsmath, mathrsfs}
\usepackage{color, tikz}

\captionsetup{
	size=small,
	labelfont=it,
	width=0.9\textwidth
}

\renewcommand{\d}{
	\ensuremath{\mathrm{d}}
}

\newcommand{\F}{
	\operatorname{\mathcal{F}}
}

\newcommand{\w}{
	\ensuremath{\omega}
}

\begin{document}

\begin{center}
\textsc{Modelska analiza I}\\
\textsc{2011/12}\\[0.5cm]
\textbf{13. naloga -- Filtriranje \v suma}\\
\end{center}
\begin{flushright}
\textbf{Jože Zobec}\\
\end{flushright}

\section{Uvod}

V tej nalogi se bomo lotili filtriranja \v suma z uporabo razli\v cnih filtrov, prvi med njimi je Wienerjev filter.
Izhodni signal je konvolucija vhodnega sigala `$u(t)$' in odzivne funkcije na\v sega "`senzorja"'\ `$v(t)$', tj.
\[
	f(t) = \int_{-\infty}^{\infty} \d\tau\ u(\tau) v(t - \tau) = (u * v)(t).
\]
Vendar pa imamo poleg $f(t)$ prime\v san \v se \v sum, tako da
\[
	c(t) = f(t) + n(t)
\]
Naj bo $\F(\bullet) = \hat{\bullet}(\w)$ fourierova transformiranka, potem velja
\begin{equation}
	\hat{c} = \hat{f} + \hat{n}, \qquad \hat{f}(\w) = \hat{u}(\w)\cdot\hat{v}(\w).
\end{equation}
Osnovna predpostavka je
\begin{equation}
	\Phi \hat{c} \approx \hat{f},
\end{equation}
oz. poiskati $\Phi(\w)$, da bo $n(t)$ minimalen. S $\Phi$ mno\v zimo $\hat{c}$ pred dekonvolucijo. Na koncu
dobimo, da je
\begin{equation}
	u(t) \approx -\frac{1}{2\pi}\F\bigg(\Phi(\w) \frac{\hat{c}(\w)}{\hat{v}(\w)}\bigg),
\end{equation}
kjer je multiplikativna konstanta $-1/2\pi$ spredaj zato, ker dvakrat naredimo Fourierovo transformacijo "`naprej"'.
Wienerjev filter $\Phi(\w)$ je definiran kot
\begin{equation}
	\Phi(\w) \equiv \frac{|\hat{f}(\w)|^2}{|\hat{f}(\w)|^2 + |\hat{n}(\w)|^2} = \frac{1}{1 +
		\displaystyle{\bigg|\frac{\hat{n}(\w)}{\hat{f}(\w)}\bigg|}^2},
\end{equation}
torej bolj za\v sumljene podatke manj upo\v stevamo v rekonstrukciji. Tu naletimo
\v se na en problem -- ne vemo, kak\v sen je \v sum. Literatura priporo\v ca oceno \v suma $n(t)$ z opazovanjem $c(t)$
po dolgem\v casu. Pri\v cakujemo, da na koncu pre\v zivi le \v sum, ki ga potem ekstrapoliramo na kratke \v case in
dolo\v cimo tudi $f(t)$.

Poleg filtra $\Phi(\w)$ lahko uporabimo tudi \v se kak\v sno okno $H(\w)$, s katerim lahko sliko dodatno izostrimo,
meglimo itd.
\begin{equation}
	u(t) = \F^{-1}\bigg(H(\w)\Phi(\w) \frac{\hat{c}(\w)}{\hat{v}(\w)}\bigg).
\end{equation}

Kompresija slik s pomo\v cjo razcepa na singularne vrednosti mo\v cno spominja na "`trunkiranje"' iz TEBD DMRG in
je precej intuitivno. Na\v sa slika je matrika $A = USV^\dagger$, $A \in \mathrm{Mat}\ n\times m$. Matrika $S$ je
diagonalna matrika singularnih vrednosti. Majhne singularne vrednosti pore\v zemo pro\v c, popravimo \v stevilo
vrstic matrike $V^\dagger$ in \v stevilo stolpcev matrike $U$. Ko zmno\v zimo nazaj, dobimo spet matriko dimenzije
$n \times m$, vendar smo za opis potrebovali manj informacije, zaradi \v cesar potrebujemo manj ra\v cunalni\v skega
spomina.

\section{Rezultati}

Za re\v sevanje sem uporabljal programski jezik {\tt C}, skriptni jezik {\tt sh} in {\tt Octave}.

\subsection{Dekonvolucija signalov}
Za datoteke {\tt signal\{1,2,3\}.dat} sem predpostavil, da je {\tt signal0.dat} brez \v suma in tako
dobil $n(t)$ eksplicitno ven, kot razliko. Zdelo se mi je tudi, da je \v cisti signal \v se vedno nekoliko
za\v sumljen zaradi efektov diskretne Fourierjeve transformacije, zato sem poleg Wienerjevega filtra dodal \v se
Hammingovo okno, ki se je izkazalo da najbolje s\v cisti oscilacije na robovih.

\begin{figure}[H]\centering
	\input{res-signal0.tex}
	\caption{Signal je tukaj videti res lep, vidimo 4 impulze. Ugibam, da gre za simuliranje
		diskretne Diracove $\delta$-funkcije, in da imajo vsi \v stirje impulzi isto
		plo\v s\v cino.}
\end{figure}

\begin{figure}[H]\centering
	\input{res-signal1.tex}
	\caption{Signal je nekoliko za\v sumljen, vendar \v se vedno vidimo jasno sliko.}
\end{figure}

\begin{figure}[H]\centering
	\input{res-signal2.tex}
	\caption{Anomalije med impulzi postajajo \v cedalje ve\v cje, vendar jih \v se vedno lahko
		lo\v cimo od impulzov.}
\end{figure}

\begin{figure}[H]\centering
	\input{res-signal3.tex}
	\caption{Ta slika je preve\v c za\v sumljena, vendar smo kljub temu izlu\v s\v cili vsaj bistveno informacijo,
		vendar bi te\v zko rekli, koliko sunkov je bilo na za\v cetku.}
\end{figure}

\v Suma smo se znebili po goljufiji. Kaj pa, \v ce ne bi imeli take sre\v ce? Kako bi potem ocenili\v sum?
Vidimo, da funkcija na za\v cetku in na koncu prav prikladno pada proti ni\v c in je skoraj konstantna. Od tam
preberemo amplitudo \v suma. Predpostavimo, da gre za beli \v sum, od koder lahko napovemo spekter
$\hat{n}(\w) \approx 1/A$, kjer je $A$ amplituda, ki smo jo merili prej. Ker je napovedan $n(\w)$ zgolj
konstanta, ta postopek ni najbolj\v si in tudi grafi, ki sem jih dobil niso jasni, zato jih ne bom kazal
tukaj.

\subsection{\v Ci\v s\v cenje Lincolnove brade}

Slike biv\v sega predsednika severnoameri\v ske federacije so bile za\v sumljene z belim \v sumom poleg tega,
da so bile zamazane z eksponentno prenosno funkcijo $v(t) = \exp(-t/\tau)/\tau$, $\tau = 30$.

Dobljenim slikam sem, tako kot prej, s Hammingovim oknom dodatno s\v cistil \v crte, ki so se poznale po
dekonvoluciji z Wienerjevim filtrom. Rezultati so prikazani na slikah~\ref{Sl1}, \ref{Sl2}, \ref{Sl3} in~\ref{Sl4}.

\begin{figure}[H]\centering
	\begin{subfigure}[b]{0.3\textwidth}
		\includegraphics[width=\textwidth, keepaspectratio=1]{lincoln_L30_N00}
		\caption{Originalna slika.}
		\label{sl11}
	\end{subfigure}
	\begin{subfigure}[b]{0.3\textwidth}
		\includegraphics[width=\textwidth, keepaspectratio=1]{fixed-lincoln_L30_N00-filter0}
		\caption{Dekonvolucija brez filtra.}
		\label{sl12}
	\end{subfigure}
	\begin{subfigure}[b]{0.3\textwidth}
		\includegraphics[width=\textwidth, keepaspectratio=1]{fixed-lincoln_L30_N00-filter1}
		\caption{Dodan filter.}
		\label{sl13}
	\end{subfigure}
	\caption{Na sliki imamo prikazan postopek \v ci\v s\v cenja slike {\tt lincoln\_L30\_N00.pgm}. Filter,
		dodan v sliki~\ref{sl13}, je Wienerjev filter, ki mu je dodano Hammingovo okno, zato je slika
		nekoliko temnej\v sa.}
	\label{Sl1}
\end{figure}

\begin{figure}[H]\centering
	\begin{subfigure}[b]{0.3\textwidth}
		\includegraphics[width=\textwidth, keepaspectratio=1]{lincoln_L30_N10}
		\caption{Originalna slika.}
		\label{sl21}
	\end{subfigure}
	\begin{subfigure}[b]{0.3\textwidth}
		\includegraphics[width=\textwidth, keepaspectratio=1]{fixed-lincoln_L30_N10-filter0}
		\caption{Dekonvolucija brez filtra.}
		\label{sl22}
	\end{subfigure}
	\begin{subfigure}[b]{0.3\textwidth}
		\includegraphics[width=\textwidth, keepaspectratio=1]{fixed-lincoln_L30_N10-filter1}
		\caption{Dodan filter.}
		\label{sl23}
	\end{subfigure}
	\caption{Na sliki imamo prikazan postopek \v ci\v s\v cenja slike {\tt lincoln\_L30\_N10.pgm}. Vidimo, da je kljub
		relativno majhnemu originalnemu \v sumu, dekonvoluirana slika mo\v cno za\v sumljena. Uporaba Wienerjevega
		filtra sliko mo\v cno izostri, defektov zaradi enodimenzionalnega branja slike skoraj ni ve\v c, ker smo
		dodali \v se Hammingovo okno.}
	\label{Sl2}
\end{figure}

\begin{figure}[H]\centering
	\begin{subfigure}[b]{0.3\textwidth}
		\includegraphics[width=\textwidth, keepaspectratio=1]{lincoln_L30_N30}
		\caption{Originalna slika.}
		\label{sl31}
	\end{subfigure}
	\begin{subfigure}[b]{0.3\textwidth}
		\includegraphics[width=\textwidth, keepaspectratio=1]{fixed-lincoln_L30_N30-filter0}
		\caption{Dekonvolucija brez filtra.}
		\label{sl32}
	\end{subfigure}
	\begin{subfigure}[b]{0.3\textwidth}
		\includegraphics[width=\textwidth, keepaspectratio=1]{fixed-lincoln_L30_N30-filter1}
		\caption{Dodan filter.}
		\label{sl33}
	\end{subfigure}
	\caption{Na sliki imamo prikazan postopek \v ci\v s\v cenja slike {\tt lincoln\_L30\_N30.pgm}. \v Sum je mo\v cno
		prisoten, vendar je kon\v cna slika \v se vedno dovolj jasna. Defekti zaradi vrsti\v cnega branja so bolj
		izraziti.}
	\label{Sl3}
\end{figure}

\begin{figure}[H]\centering
	\begin{subfigure}[b]{0.3\textwidth}
		\includegraphics[width=\textwidth, keepaspectratio=1]{lincoln_L30_N40}
		\caption{Originalna slika.}
		\label{sl41}
	\end{subfigure}
	\begin{subfigure}[b]{0.3\textwidth}
		\includegraphics[width=\textwidth, keepaspectratio=1]{fixed-lincoln_L30_N40-filter0}
		\caption{Dekonvolucija brez filtra.}
		\label{sl42}
	\end{subfigure}
	\begin{subfigure}[b]{0.3\textwidth}
		\includegraphics[width=\textwidth, keepaspectratio=1]{fixed-lincoln_L30_N40-filter1}
		\caption{Dodan filter.}
		\label{sl43}
	\end{subfigure}
	\caption{Na sliki imamo prikazan postopek \v ci\v s\v cenja slike {\tt lincoln\_L30\_N40.pgm}. Slika~\ref{sl42} je
	prakti\v cno nerazpoznavna, po uporabi filtra, na sliki~\ref{sl43}, pa je tako jasna, da skorajda ni mo\v c verjeti
	kako mo\v cno orodje so filtri.}
	\label{Sl4}
\end{figure}

\subsection{Kompresija slik z uporabo SVD}

Obdr\v zimo $M$ singularnih vrednosti in originalno matriko aproksimiramo s tako, ki ima isto dimenzijo, a ima ni\v zji rang
(rang $M$). Rezultata za sliki {\tt dexter.pgm} in {\tt a.pgm} sta na slikah~\ref{dexter} in~\ref{Sa}.

\begin{figure}[H]\centering
	\begin{subfigure}[b]{0.3\textwidth}
		\includegraphics[width=\textwidth, keepaspectratio=1]{dexter}
		\caption{{\tt dexter.pgm}}
		\label{dexter1}
	\end{subfigure}
	\begin{subfigure}[b]{0.3\textwidth}
		\includegraphics[width=\textwidth, keepaspectratio=1]{small-dexter-M30}
		\caption{$M = 30$.}
		\label{dexter2}
	\end{subfigure}
	\begin{subfigure}[b]{0.3\textwidth}
		\includegraphics[width=\textwidth, keepaspectratio=1]{small-dexter-M20}
		\caption{$M = 20$.}
		\label{dexter3}
	\end{subfigure}
	\caption{Na sliki imamo prikazanih nekaj razli\v cnih izbir $M$ za sliko {\tt dexter.pgm}. Obe izbiri $M = 30$ in
		$M = 20$ sta \v se dobri, vendar na $M = 20$ \v ze te\v zko lo\v cimo Dexterjevo privzdignjeno obrv.}
	\label{dexter}
\end{figure}

\begin{figure}[H]\centering
	\begin{subfigure}[b]{0.18\textwidth}
		\includegraphics[width=\textwidth, keepaspectratio=1]{a}
		\caption{{\tt a.pgm}}
		\label{Sa1}
	\end{subfigure}
	\begin{subfigure}[b]{0.18\textwidth}
		\includegraphics[width=\textwidth, keepaspectratio=1]{small-a-M30}
		\caption{$M = 30$.}
		\label{Sa2}
	\end{subfigure}
	\begin{subfigure}[b]{0.18\textwidth}
		\includegraphics[width=\textwidth, keepaspectratio=1]{small-a-M20}
		\caption{$M = 20$.}
		\label{Sa3}
	\end{subfigure}
	\begin{subfigure}[b]{0.18\textwidth}
		\includegraphics[width=\textwidth, keepaspectratio=1]{small-a-M10}
		\caption{$M = 10$.}
		\label{Sa4}
	\end{subfigure}
	\begin{subfigure}[b]{0.18\textwidth}
		\includegraphics[width=\textwidth, keepaspectratio=1]{small-a-M8}
		\caption{$M = 8$.}
		\label{Sa5}
	\end{subfigure}
	\caption{Razli\v cne izbire kompresije prek SVD za sliko {\tt a.pgm}.}
	\label{Sa}
\end{figure}

S preprostim ukazom\footnote{{\tt \$ du {}-{}-apparent-size ime\_datoteke}} lahko \emph{ocenimo}
u\v cinkovitost na\v se kompresije s singularnim razcepom. Rezultati so v tabeli~\ref{tab1}.
\begin{table}[H]\centering
	\caption{Ocena velikosti datotek {\tt dexter} po kompresiji z SVD. Vse datoteke so
	pretvorjene v format {\tt .png}. Izvirnik v formatu {\tt .png} potrebuje samo
	14 kiB pomnilnika. Po spodnjih \v stevilih sode\v c SVD kompresija verjetno potrebuje
 	kak dodaten korak, saj na\v s rokodelski poskus kompresije klavrno propade in dose\v ze
	ravno obratno -- pomnilnik, potreben za te datoteke, se \emph{pove\v ca} za ve\v c
	kot $100\%$.}
	\begin{tabular}{r|l}
		$M$ & velikost [kiB] \\
		\hline
		2  & 32 \\
		8  & 37 \\
		10 & 37 \\
		20 & 36 \\
		30 & 36 \\
		39 & 36 \\
		50 & 36 \\
		60 & 36 \\
		80 & 36 \\
		100& 35
	\end{tabular}
	\label{tab1}
\end{table}
Vidimo, da se taka kompresija ne izpla\v ca preve\v c. Tudi sicer obstajajo naprednej\v se metode.

\section{Zaklju\v cek}

Tekom te naloge smo se pobli\v ze seznanili s pomembnosmi filtrov in preproste kompresije s singularnim
razcepom, ki v praksi verjetno ni najbolj\v si pristop.

\end{document}
