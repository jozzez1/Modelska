\documentclass[a4 paper, 12pt]{article}
\usepackage[slovene]{babel}
\usepackage[utf8]{inputenc}
\usepackage[T1]{fontenc}
\usepackage[small, width=0.9\textwidth,labelfont=it]{caption}
\usepackage[pdftex]{graphicx}
\usepackage[usenames, dvipsnames]{xcolor}
\usepackage{amssymb, fullpage, float, pdflscape, subcaption, amsmath, color, hyperref}

\renewcommand{\d}{
	\ensuremath{\mathrm{d}}
}

\newcommand{\e}{
	\ensuremath{\mathrm{e}}
}

\renewcommand{\r}{
	\ensuremath{\mathbf{r}}
}

\hypersetup{
	colorlinks=true,
	linkcolor=black!60!red,
	citecolor=black!60!green,
	urlcolor=black!60!cyan
}

\begin{document}

\begin{center}
\textsc{Modelska analiza II}\\
\textsc{2011/12}\\[0.5cm]
\textbf{9. naloga -- Metoda robnih elementov}
\end{center}
\begin{flushright}
\textbf{Jože Zobec}\\
\end{flushright}

\section{Uvod}

Metoda robnih elementov je intimno povezana z metodo kon\v cnih elementov. V problemih dimenzije tri se
obe metodi zana\v sata na triangulacijo, vendar metodo kon\v cnih elementov zanima obna\v sanje znotraj telesa,
z metodo robnih elementov pa obna\v sanje zunaj telesa, ki ima rob obmo\v cja $\mathcal{A}$, nad katerim
je definirana na\v sa ena\v cba. V dveh dimenzijah so robni elementi zgolj intervali, tako kot kon\v cni
elementi v eni dimenziji. To je zato, ker v ketodi kon\v cnih elementov zado\v s\v ca zgolj opis
$\partial \mathcal{A}$, ki pa ima eno dimenzijo manj, kot $\mathcal{A}$.

Rob torej razre\v zemo na robne elemente, kjer lahko izra\v cunamo prispevek kateregakoli elementa na
katerokoli to\v cko v prostoru. Tj. $u^{(j)}(\r)$ je prispevek $j$-tega elementa na to\v cko $\r$ (definicija
je v navodilu). Medsebojne prispevke robnih elementov zlo\v zimo v matriko $\mathbf{A}$, ki ima komponente
\[
	A_{ij} = u^{(j)}(\r_i),
\]
kjer vektor $\r_i$ ka\v ze v te\v zi\v s\v ce $i$-tega robnega elementa.
Ta matrika je polna in simetri\v cna. Sicer v splo\v snem potrebujemo \v se eno matriko, vendar ne v
tem konkretnem primeru. Na koncu moramo re\v siti sistem $N$ linearnih ena\v cb,
\[
	A_{ij} \sigma_j = \phi_i = 1, \quad \forall i.
\] 
Tako dobimo $\sigma_j$, ki je porazdelitev naboja na robu. Potencial v to\v cki $\r$ dobimo kot
\[
	\phi(\r) = \sum_{i = 1}^N \sigma_i u^{(i)} (\r).
\] 

Ko re\v sujemo obtok teko\v cine ob krilu nam ni treba re\v sevati sistema ena\v cb, vendar, \v ce ho\v cemo
dobiti tokovnice moramo re\v siti ena\v cbo diferencialno ena\v cbo
\[
	\begin{bmatrix} \dot{x} \\ \dot{y}\end{bmatrix} = \begin{bmatrix} v_x \\ v_y \end{bmatrix},
\]
ki jo lahko re\v simo npr. s konstrukcijo simplekti\v cnega integratorja s Trotter-Suzukijevim razcepom
(komutirajo\v ca opratorja sta v tem primeru $T_x$ in $T_y$, torej kineti\v cni energiji posameznih
komponent $\r$), tj.
\begin{align*}
	\exp\big(c\tau\{\bullet, T_x\}\big) \begin{bmatrix} x \\ y \end{bmatrix} &= \begin{bmatrix} x
		+ c\tau v_x (x,y) \\ y \end{bmatrix} \\
	\exp\big(c\tau\{\bullet, T_y\}\big) \begin{bmatrix} x \\ y \end{bmatrix} &= \begin{bmatrix} x \\ y
		+ c\tau v_y (x, y) \end{bmatrix}
\end{align*}
od koder lahko uporabimo katerokoli shemo ho\v cemo, npr. $S_2$, ki je najbolj preprosta in je reda
$\mathcal{O}(\tau^2)$, zapi\v se pa se kot
\[
	\begin{bmatrix}
		x \\ y
	\end{bmatrix}_{t + \tau} = \begin{bmatrix}
		x + \frac{\tau}{2} v_x (x,y) + \frac{\tau}{2}v_x\big(x + \frac{\tau}{2} v_x(x,y), y +
			\tau v_y (x,y)\big) \\
		y + \tau v_y (x,y)
	\end{bmatrix}_t,
\]
hitrosti pa sta
\begin{align*}
	v_x (x,y) &= \sum_{i = 1}^N \big(t_x^{(i)} v_\parallel^{(i)}(x,y) + t_y^{(i)}v_\perp^{(i)}(x,y)\big), \\
	v_y (x,y) &= \sum_{i = 1}^N \big(-t_x^{(i)} v_\parallel^{(i)}(x,y) + t_y^{(i)}v_\perp^{(i)}(x,y)\big),
\end{align*}
kjer je $\mathbf{t}^{(i)}$ enotski smerni vektor ustreznega robnega elementa,
$\mathbf{t}^{(i)} \propto \r_{i+1} - \r_i$. Analogno lahko preklapljamo med lokalnimi in globalnimi
koordinatami z rotacijsko matriko
\[
	\r^{(i)} = \begin{bmatrix} x' \\ y' \end{bmatrix}^{(i)} = \begin{bmatrix} t_x^{(i)} x - t_y^{(i)}y \\
		t_y^{(i)} x + t_x^{(i)} y \end{bmatrix}
\]

Literatura~\cite[str.~659]{sirca} sicer navaja, da je v splo\v snem potrebna \v se ena matrika, ki problem malo
ote\v zi, vendar zaupam poenostavitvi iz navodila naloge.
\section{Rezultati}

Za ra\v cunanje in prikaz rezultatov sem uporabil {\tt C++} s knji\v znicama {\tt MathGL} in {\tt Eigen} in
prevajalnik {\tt clang++}. Porazdelitev naboja na traku, $\sigma(x)$, je na sliki~\ref{gr1}, potencial $\phi(\r)$ pa je
na sliki~\ref{gr2}.

\begin{figure}[H]\centering
	\includegraphics[width=0.8\textwidth]{naboj}
	\caption{Na skrajnih robovih je gostota naboja vi\v sja, kot v sredini, kar lahko
		pri\v cakujemo zato, ker se elektri\v cni naboj z ostim znakom odbija.}
	\label{gr1}
\end{figure}

\begin{figure}[H]\centering
	\includegraphics[width=0.8\textwidth]{potencial}
	\caption{Na tej sliki imamo graf z elektri\v cnim potencialom v okolici traku. Pri\v cakujemo
		divergenco, kjer je naboj, kar res dobimo.}
	\label{gr2}
\end{figure}

\subsection{Obtekanje krila}

Za obtekanje krila imamo v bistvu la\v zji problem, saj moramo zgolj izra\v cunati hitrostno polje, ki je
gradient potenciala.

\begin{figure}[H]\centering
	\includegraphics[width=0.8\textwidth]{vektor}
	\caption{Vektorsko polje za elipsoid.}
	\label{gr3}
\end{figure}

\begin{thebibliography}{9}
	\bibitem{sirca}
		S. \v Sirca in M. Horvat,
		{\em Ra\v cunske metode za fizike},
		DMFA Zalo\v zni\v stvo,
		(2010)
\end{thebibliography}

\end{document}
