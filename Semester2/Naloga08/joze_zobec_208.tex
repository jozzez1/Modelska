\documentclass[a4 paper, 12pt]{article}
\usepackage[slovene]{babel}
\usepackage[utf8]{inputenc}
\usepackage[T1]{fontenc}
\usepackage[small, width=0.9\textwidth,labelfont=it]{caption}
\usepackage[pdftex]{graphicx}
\usepackage[usenames, dvipsnames]{xcolor}
\usepackage{amssymb, fullpage, float, pdflscape, subcaption, amsmath, color, hyperref}

\renewcommand{\d}{
	\ensuremath{\mathrm{d}}
}

\newcommand{\e}{
	\ensuremath{\mathrm{e}}
}

\renewcommand{\r}{
	\ensuremath{\mathbf{r}}
}

\hypersetup{
	colorlinks=true,
	linkcolor=black!60!red,
	citecolor=black!60!green,
	urlcolor=black!60!cyan
}

\begin{document}

\begin{center}
\textsc{Modelska analiza II}\\
\textsc{2011/12}\\[0.5cm]
\textbf{8. naloga -- Metoda kon\v cnih elementov: lastne re\v sitve}
\end{center}
\begin{flushright}
\textbf{Jože Zobec}\\
\end{flushright}

\section{Uvod}

Naloga je analogna prej\v snji. Globalna matrika $A$ je popolnoma ista, imamo le \v se dodatno matriko
$B$. Nato re\v simo posplo\v sen problem lasnih vrednosti:
\[
	(A - \lambda B) x = 0,
\]
oz.
\[
	(B^{-1}A - \lambda I) x = 0.
\]
Lastni pari so energije in lastni nihajni na\v cini, zaradi \v cesar so vse lastne vrednosti realne
(sicer pa je iz definicije $A$ in $B$ o\v citno, da imamo opravka z realnima simetri\v cnima
matrikama).

Galerkinov nastavek lahko re\v sujemo na popolnoma enak na\v cin, le da tam triangulacija prostora
ni potrebna, saj morajo bazne funkcije $\psi^m_i$ biti izbrane tako, da \v ze same po sebi zadostijo
robnim pogojem, tj.
\[
	u (r, \varphi) \approx \sum_{m, i} c_{m,i} \psi^m_i (r, \varphi), \qquad \psi^m_i (r, \varphi)
		\equiv r^{m + i} (r - 1) \sin (m\varphi)
\]
V tem primeru veljata isti definiciji matrik $A$ in $B$, kot prej, le da funkcije $\psi^m_i$ niso tako
lokalizirane, se pravi
\[
	A^{(m)}_{ij} \equiv \langle \nabla \psi^m_i, \nabla \psi^m_j \rangle, \quad
	B^{(m)}_{ij} \equiv \langle \psi^m_i, \psi^m_j \rangle,
\]
s skalarnim produktom $\langle \bullet, \bullet \rangle$ nad obmo\v cjem $\mathcal{A}$
\[
	\langle f, g \rangle \equiv \int_\mathcal{A} \d S\ f(x,y) g(x,y) =
		\int_\mathcal{A} f(r,\varphi) g(r, \varphi) r \d r\ \d\varphi,
\]
ki je obmo\v cje nad katerim velja diferencialna ena\v cba $\nabla^2 u + k^2 u = 0$, z
Dirichletovimi robnimi pogoji na robu, tj. $u (r,\varphi) = 0, \forall (r, \varphi) \in
\partial\mathcal{A}$.

Matriki $A$ in $B$ sta potem po definiciji
\begin{align}
	B^{(m)}_{ij} &= \frac{\pi}{2}\bigg[\frac{1}{2m + i + j + 2} - \frac{2}{2m + i + j + 3} +
		\frac{1}{2m + i + j + 4}\bigg] \\
	A^{(m)}_{ij} &= \frac{\pi}{2} \bigg[\frac{ij}{2m + i + j} - \frac{2ij + i + j}{2m + i + j + 1} +
		\frac{(i + 1)(j + 1)}{2m + i + j + 2}\bigg]
\end{align}

Ta problem je glede na prej\v snjo nalogo ra\v cunsko zahtevnej\v si (zahteva $\mathcal{O}(N^3)$
operacij), vendar si lahko pomagamo s tem, da ne potrebujemo vseh lastnih vrednosti, pa\v c pa samo
npr. nekaj najmanj\v sih.

\section{Rezultati}

Triangulacijo sem izra\v cunal s programskim jezikom {\tt C}, Galerkonov nastavek pa kar z {\tt Octave}.
Nekaj lastnih na\v cinov sem prikazal na graf (triangulacijo sem prikazal z {\tt Octave}). Nekaj
takih lasnih re\v sitev lahko vidimo na slikah~\ref{gr1}, \ref{gr2}, \ref{gr3} in \ref{gr4}. Energije so tabelirane v
tabeli~\ref{tab1}.

\begin{figure}[H]\centering
	\includegraphics[width=0.8\textwidth, keepaspectratio=1]{mode-0}
	\caption{Osnovno stanje polkro\v zne opne, $n = 1$, $m = 1$.}
	\label{gr1}
\end{figure}

\begin{figure}[H]\centering
	\includegraphics[width=0.8\textwidth, keepaspectratio=1]{mode-1}
	\caption{Prvo vzbujeno stanje, $n = 1$, $m = 2$.}
	\label{gr2}
\end{figure}

\begin{figure}[H]\centering
	\includegraphics[width=0.8\textwidth, keepaspectratio=1]{mode-2}
	\caption{Drugo vzbujeno stanje, $n = 1$, $m = 3$.}
	\label{gr3}
\end{figure}

\begin{figure}[H]\centering
	\includegraphics[width=0.8\textwidth, keepaspectratio=1]{mode-3}
	\caption{Tretje vzbujeno stanje, $n = 2$, $m = 1$.}
	\label{gr4}
\end{figure}

\begin{figure}[H]\centering
	\includegraphics[width=0.8\textwidth, keepaspectratio=1]{mode-4}
	\caption{\v Cetrto vzbujeno stanje polkro\v zne opne, $n = 1$, $m = 4$.}
	\label{gr5}
\end{figure}

\begin{figure}[H]\centering
	\includegraphics[width=0.8\textwidth, keepaspectratio=1]{mode-5}
	\caption{Peto vzbujeno stanje, $n = 2$, $m = 2$.}
	\label{gr6}
\end{figure}

\begin{figure}[H]\centering
	\includegraphics[width=0.8\textwidth, keepaspectratio=1]{mode-6}
	\caption{\v Sesto vzbujeno stanje, $n = 1$, $m = 5$.}
	\label{gr7}
\end{figure}

\begin{figure}[H]\centering
	\includegraphics[width=0.8\textwidth, keepaspectratio=1]{mode-7}
	\caption{Sedmo vzbujeno stanje, $n = 2$, $m = 3$.}
	\label{gr8}
\end{figure}

\begin{figure}[H]\centering
	\includegraphics[width=0.8\textwidth, keepaspectratio=1]{mode-8}
	\caption{Osmo vzbujeno stanje polkro\v zne opne, $n = 1$, $m = 6$.}
	\label{gr9}
\end{figure}

\begin{figure}[H]\centering
	\includegraphics[width=0.8\textwidth, keepaspectratio=1]{mode-9}
	\caption{Deveto vzbujeno stanje, $n = 3$, $m = 1$.}
	\label{gr10}
\end{figure}

\begin{figure}[H]\centering
	\includegraphics[width=0.8\textwidth, keepaspectratio=1]{mode-10}
	\caption{Deseto vzbujeno stanje, $n = 2$, $m = 4$.}
	\label{gr11}
\end{figure}

\begin{figure}[H]\centering
	\includegraphics[width=0.8\textwidth, keepaspectratio=1]{mode-11}
	\caption{Enajsto vzbujeno stanje, $n = 1$, $m = 7$.}
	\label{gr12}
\end{figure}


\begin{table}[H]\centering
	\caption{Triangulacije so bila izra\v cunane z $\sim 12 \cdot 10^3$ to\v ckami in dajejo rezultate
		na slabi dve decimalni mesti. Vendar pa nam dajo rezultate na bistveno \v sir\v sem spektru.
		Te lastne vrednosti, kar nam jih je dal Galerkin so vse, kar se jih je dalo nara\v cunati,
		sicer matrika ni bila ve\v c pozitivno definitna, vendar pa da ob\v casno napove vse \v stiri
		decimalke pravilno. Moj sklep je, \v ce \v zelimo veliko lastnih vrednosti na majhno natan\v cnost
		potrebujemo kon\v cne elemente, sicer pa kar Galerkina, ki je hitrej\v si.}
	\begin{tabular}{r|c|c|l}
		$n$ & $E_n^\text{TRI}$ & $E_n^\text{G}$ & $E_n$ \\
		\hline
		0 &	3.8327  & 3.8337 & 3.8317 \\
		1 &	5.1379  & 5.1356 & 5.1356 \\
		2 &	6.3846  & 6.3802 & 6.3802 \\
		3 &	7.0216  & 7.0219 & 7.0156 \\
		4 &	7.5959  & 7.5884 & 7.5883 \\
		5 &	8.4275  & 8.4186 & 8.4127 \\
		6 &	8.7832  & 8.7715 & 8.7715 \\
		7 &	9.7770  & 9.7611 & 9.7610 \\
	\end{tabular}
	\label{tab1}
\end{table}

\begin{thebibliography}{9}
	\bibitem{sirca}
		S. \v Sirca in M. Horvat,
		{\em Ra\v cunske metode za fizike},
		DMFA Zalo\v zni\v stvo,
		(2010)
\end{thebibliography}

\end{document}
