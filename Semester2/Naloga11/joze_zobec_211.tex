\documentclass[a4 paper, 12pt]{article}
\usepackage[slovene]{babel}
\usepackage[utf8]{inputenc}
\usepackage[T1]{fontenc}
\usepackage[small, width=0.7\textwidth,labelfont=it]{caption}
\usepackage[pdftex]{graphicx}
\usepackage[usenames, dvipsnames]{xcolor}
\usepackage{amssymb, fullpage, float, pdflscape, subcaption, amsmath, color, hyperref}

\hypersetup{
	colorlinks=true,
	linkcolor=black!60!red,
	citecolor=black!60!green,
	urlcolor=black!60!cyan
}

\newcommand{\vfi}{
    \ensuremath{\varphi}
}

\renewcommand{\d}{
    \ensuremath{\mathrm{d}}
}

\begin{document}

\begin{center}
\textsc{Modelska analiza II}\\
\textsc{2011/12}\\[0.5cm]
\textbf{11. naloga -- Gibanje neraztegljive vrvice}
\end{center}
\begin{flushright}
\textbf{Jože Zobec}\\
\end{flushright}

\section{Razmislek}

V tej nalogi bomo simulirali gibanje neraztegljive vrvice. Imamo dve ena\v cbi in dve neznanki:
\begin{align}
    F \bigg(\frac{\partial\varphi}{\partial s}\bigg)^2 - \frac{\partial^2 F}{\partial s^2} &=
        \bigg(\frac{\partial\varphi}{\partial t}\bigg)^2 \\
    F\frac{\partial^2\varphi}{\partial s^2} + 2\frac{\partial F}{\partial s}\frac{\partial \varphi}{\partial s} &=
        \frac{\partial^2 \varphi}{\partial t^2}
\end{align}
Imamo za\v cetni pogoj $\varphi (s,t = 0) = \varphi_0 (s) = \text{konst}$. V prijemali\v s\v cu vrvice
velja robni pogoj
\[
    F \frac{\partial \varphi}{\partial s} + \cos \varphi = 0,
        \qquad \frac{\partial F}{\partial s} + \sin \varphi = 0,
\]
na odprtem koncu pa je
\[
    F (s = 1) = 0, \qquad \frac{\partial^2 \varphi}{\partial s^2} = 0.
\]
Da lahko re\v simo prvo ena\v cbo potrebujemo \v se $\dot{\varphi}(s,t = 0)$. Tu uporabimo namig,
da vrvica na za\v cetku miruje, tj. $\dot{\varphi}(s, t = 0) = 0$. Sedaj poznamo vse.

Nalogo bomo re\v sevali z diskretnimi pribli\v zki -- vrvico aproksimiramo z odsekoma ravno krivuljo,
kjer so vozli\v s\v ca razmaknjena za $h = 1/N$.

Diferen\v cne ena\v cbe so so reda $\mathcal{O}(h)$ in se glasijo:
\begin{align}
    F^t_{i-1} + F^t_{i+1} - F^t_i\Bigg[2 + \bigg(\frac{\vfi^t_{i+1} - \vfi^t_{i-1}}{2}\bigg)^2\Bigg] =
        -\frac{h^2}{h_t^2}\big(\vfi^t_i - \vfi^{t-1}_i\big)^2,
        \label{prva} \\
    \vfi^{t+1}_i = \frac{h_t^2}{h^2}\Bigg[\frac{1}{2}\big(\vfi^t_{i+1} - \vfi^t_{i-1}\big)
        \big(F^t_{i+1} - F^t_{i-1}\big) + F^t_i\big(\vfi^t_{i+1} + \vfi^t_{i-1} - 2\vfi^t_i\big)\Bigg] +
        2\vfi^t_i - \vfi^{t-1}_i,
        \label{druga}
\end{align}
na koncéh za silo velja
\begin{align}
    F^t_1 &= h\sin\vfi_1^t - F^t_2, 
    \label{F1} \\
    F^t_N &= 0,
    \label{FN}
\end{align}
kar moramo v ena\v cbi~\eqref{prva} upo\v stevati kot posebno ena\v cbo za $F_2$:
\begin{equation}
    F^t_3 - F_2^t\Big[1 + \textstyle{\frac{1}{4}}(\vfi^t_3 - \vfi^t_1)\Big] = -\big(\textstyle{\frac{h}{h_t}}\big)^2
        (\vfi_2^t - \vfi_2^{t-1}) - h\sin\vfi^t_1.
    \label{rp}
\end{equation}
Robni pogoji za kot $\vfi$ so
\begin{align}
    \vfi_1^t = \frac{2h}{F^{t-1}_2}\cos\vfi^t_2 + \vfi^t_3
    \label{f1} \\
    \vfi_N^t = 2\vfi^t_{N-1} - \vfi^t_{N-2}
    \label{fN}
\end{align}
Poznamo $\vfi^0_i = \vfi^1_i = \vfi_0$, od koder lahko izra\v cunamo vse ostalo.

\subsection{Algoritem}

Najprej iz $\vfi (s, t=1)$ izra\v cunamo $F^1 (s)$ prek ena\v cb~\eqref{prva} (z robnim pogojem~\eqref{rp}) za
$i = 2,\ldots,N-1$. $F^1_1$ in $F^1_N$ dolo\v cimo iz robnih pogojev~\eqref{F1} in~\eqref{FN}. Sedaj, ko imamo
za\v cetna $F^t_i$ in $\vfi^t_i$ izvajamo algoritem:
\begin{enumerate}
    \item{Prek ena\v cbe~\eqref{druga} izra\v cunamo $\vfi^{t+1}_i$ za $i = 2,\ldots,N-1$.}
    \item{Re\v simo ena\v cbo~\eqref{f1} in izra\v cunamo $\vfi^{t+1}_1$, iz ena\v cbe~\eqref{fN} pa
        izra\v cunamo $\vfi^{t+1}_N$.}
    \item{Re\v simo tridiagonalni sistem ena\v cb~\eqref{prva} (in~\eqref{rp} za $F^{t+1}_2$) in
        izra\v cunamo $F^{t+1}_i$ za $i = 2,\ldots,N-1$.}
    \item{Izra\v cunamo $F^{t+1}_1$ in $F^{t+1}_N$ po ena\v cbah~\eqref{F1} in~\eqref{FN}.}
    \item{Premaknemo \v casovni indeks $t \to t+1$ in se vrnemo na to\v cko 1.}
\end{enumerate}

Rezultati so mo\v cno odvisni od tega, \v ce smo pravilno upo\v stevali predznake, kar mi je povzro\v cilo nemalo
preglavic.

\section{Rezultati}

Grobo \v sablono programa sem najprej napravil v {\tt Octave}, ki pa se je izvajala prepo\v casi, zato sem
napravil za hitrost opzimizirano varianto v programskem jeziku {\tt C}, pri \v cemer sem si dodatno pomagal
s programom za profilno vodeno optimizcijo, {\tt gprof}. Kot ponavadi, sem tudi tokrat \v zrtvoval nekaj
natan\v cnosti na ra\v cun hitrosti in vklopil {\tt -ffast-math}. Prevajalnik {\tt gcc-4.7} pri tem ne daje ve\v c
konsistentnih rezultatov, {\tt clang-3.3} pa je \v se vedno stabilen pravilno zaokro\v zuje, zato sem uporabil
slednjega.

Ugotovil sem, da v moji implementaciji najve\v c klicev dobimo za izra\v cun diagonale na\v sega tridiagonalnega
sistema, \v casovno najpotratnej\v si proces pa je izra\v cun $\vfi^{t+1}$.

Rezultate sem za razli\v cne za\v cetne pogoje prikazal na prilo\v zenih animacijah, barva vzdol\v z verige
prikazuje silo $F(s,t)$ med \v clenki, vektorji pa prikazujejo hitrost in so dol\v zine $2\sqrt{v_x^2 + v_y^2}$.
Izris sem napravil s knji\v znico {\tt MathGL}, animacijo pa s programom {\tt mencoder}.
\begin{itemize}
    \item{{\tt ravna.avi} upo\v steva za\v cetne pogoje, kot jih navaja navodilio, veriga je videti nekam toga.
        Za\v cetni kot sem nastavil na $10\pi/3$.}
    \item{{\tt opletanje.avi}, tu sem za za\v cetne pogoje vzel $\vfi^1_i = \sin(i\pi/2N)$, za
        $\vfi^0_i = \vfi^1_i + h_t i/N$. Opazoval sem opletanje verige. Ugotovil sem, da model vozlov na verigi ne
        opi\v se pravilno -- v veliko primerih se takrat veriga strga (tj. $F \to \infty$), kar pa, kot vemo iz lastnih
        izku\v senj, se ne zgodi prav pogosto. Zelo lepo je na za\v cetku videti povratni udarec verige ("`backlash"'),
        vendar pa zaradi vozlov model na koncu dinamike ne opi\v se realisti\v cno.}
    \item{{\tt valovanje.avi} prikazuje kako lepo se valovi \v sirijo vzdol\v z verige, venda pa eden izmed vozlov verigo
        spet na koncu strga. $\vfi^1_i$ je isti kot za datoteko {\tt opletanje.avi}, vendar pa je
        $\vfi^0_i = \vfi^1_i + h_t\sin(i\pi/N)$.}
    \item{{\tt cela.avi} prikazuje $\vfi^0_i = \vfi^1_i = \frac{\pi}{10}i/N$. Ta za\v cetni pogoj je verigo obdr\v zal
        celo dlje, zato sem to datoteko kar poimenoval {\tt cela.avi}.}
\end{itemize}
Animacije so narejene pri $h_t = 10^{-5}$, $N = 500$, zlepil sem po 34 sli\v cic na sekundo. Med izrisom dveh zaporednih
sli\v cic, sem jih $10^5$ izpustil, razen za datoteke {\tt opletanje.avi}, {\tt valovenje.avi} in {\tt cela.avi}, kjer
sem jih izpustil po 1000.

Ker imamo vse hitrosti in pozicije, lahko iz tega izra\v cunamo tudi energijo. Ohranjanje energije nam bo pokazatelj
natan\v cnosti metode.

Energija je vsota kinteti\v cne in potencialne: $E_\text{tot.} = T + V = T_\text{trans.} + T_\text{rot.} + V$.
Grag~\ref{gr1} prikazuje ohranjanje energije.

\begin{figure}[H]\centering
    \includegraphics[width=0.9\textwidth]{energija1.png}
    \caption{Ohranjanje energije za animacijo {\tt ravna.avi}. Zeleni del prikazuje kineti\v cno energijo, modri
        del pa potencialno, pri kateri sem dodal $h_0 = 0.6$, da bi zagotovil $V \geq 0$. Indeks `$t$' je
        \v casovni indeks, ki je bil inkrementiran po vsakih $h_t^{-1}$ (tj. $10^5$) korakih. Energija se ohranja
        na tri decimalna mesta.}
    \label{gr1}
\end{figure}

\end{document}
