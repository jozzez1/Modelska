\documentclass[a4 paper, 12pt]{article}
\usepackage[slovene]{babel}
\usepackage[utf8]{inputenc}
\usepackage[T1]{fontenc}
\usepackage[small, width=0.7\textwidth,labelfont=it]{caption}
\usepackage[pdftex]{graphicx}
\usepackage[usenames, dvipsnames]{xcolor}
\usepackage{amssymb, fullpage, float, pdflscape, subcaption, amsmath, color, hyperref}

\hypersetup{
	colorlinks=true,
	linkcolor=black!60!red,
	citecolor=black!60!green,
	urlcolor=black!60!cyan
}

\newcommand{\vfi}{
    \ensuremath{\varphi}
}

\begin{document}

\begin{center}
\textsc{Modelska analiza II}\\
\textsc{2011/12}\\[0.5cm]
\textbf{11. naloga -- Gibanje neraztegljive vrvice}
\end{center}
\begin{flushright}
\textbf{Jože Zobec}\\
\end{flushright}

\section{Razmislek}

V tej nalogi bomo simulirali gibanje neraztegljive vrvice. Imamo dve ena\v cbi in dve neznanki:
\begin{align}
    F \bigg(\frac{\partial\varphi}{\partial s}\bigg)^2 - \frac{\partial^2 F}{\partial s^2} &=
        \bigg(\frac{\partial\varphi}{\partial t}\bigg)^2 \\
    F\frac{\partial^2\varphi}{\partial s^2} + 2\frac{\partial F}{\partial s}\frac{\partial \varphi}{\partial s} &=
        \frac{\partial^2 \varphi}{\partial t^2}
\end{align}
Imamo za\v cetni pogoj $\varphi (s,t = 0) = \varphi_0 (s) = \text{konst}$. V prijemali\v s\v cu vrvice
velja robni pogoj
\[
    F \frac{\partial \varphi}{\partial s} + \cos \varphi = 0,
        \qquad \frac{\partial F}{\partial s} + \sin \varphi = 0,
\]
na odprtem koncu pa je
\[
    F (s = 1) = 0, \qquad \frac{\partial^2 \varphi}{\partial s^2} = 0.
\]
Da lahko re\v simo prvo ena\v cbo potrebujemo \v se $\dot{\varphi}(s,t = 0)$. Tu uporabimo namig,
da vrvica na za\v cetku miruje, tj. $\dot{\varphi}(s, t = 0) = 0$. Sedaj poznamo vse.

Nalogo bomo re\v sevali z diskretnimi pribli\v zki -- vrvico aproksimiramo z odsekoma ravno krivuljo,
kjer so vozli\v s\v ca razmaknjena za $h = 1/N$.

Diferen\v cne ena\v cbe so so reda $\mathcal{O}(h)$ in se glasijo:
\begin{align}
    F^t_{i-1} + F^t_{i+1} - F^t_i\Bigg[2 + \bigg(\frac{\vfi^t_{i+1} - \vfi^t_{i-1}}{2}\bigg)^2\Bigg] =
        \frac{h^2}{h_t^2}\big(\vfi^t_i - \vfi^{t-1}_i\big)^2,
        \label{prva} \\
    \vfi^{t+1}_i = \frac{h_t^2}{h^2}\Bigg[\frac{1}{2}\big(\vfi^t_{i+1} - \vfi^t_{i-1}\big)
        \big(F^t_{i+1} - F^t_{i-1}\big) + F^t_i\big(\vfi^t_{i+1} + \vfi^t_{i-1} - 2\vfi^t_i\big)\Bigg] +
        2\vfi^t_i - \vfi^{t-1}_i,
        \label{druga}
\end{align}
na koncéh za silo velja
\begin{align}
    F^t_1 &= h\sin\vfi_1^t - F^t_2, 
    \label{F1} \\
    F^t_N &= 0,
    \label{FN}
\end{align}
za kot $\vfi$ pa
\begin{align}
    \vfi_1^t = \frac{2h}{F^{t-1}_2}\cos\vfi^t_2 - \vfi^t_3
    \label{f1} \\
    \vfi_N^t = 2\vfi^t_{N-1} - \vfi^t_{N-2}
    \label{fN}
\end{align}
Poznamo $\vfi^0_i = \vfi^1_i = \vfi_0$, od koder lahko izra\v cunamo vse ostalo.

\subsection{Algoritem}
Ra\v cunski postopek izgleda takole:
\begin{enumerate}
    \item{Re\v simo tridiagonalni sistem ena\v cb~\eqref{prva} in izra\v cunamo $F^{t}_i$
        za $i = 2,\ldots,N-1$.}
    \item{Izra\v cunamo $F^t_1$ in $F^t_N$ po ena\v cbah~\eqref{F1} in~\eqref{FN}.}
    \item{Prek ena\v cbe~\eqref{druga} izra\v cunamo $\vfi^{t+1}_i$ za $i = 2,\ldots,N-1$.}
    \item{Re\v simo ena\v cbo~\eqref{f1} in izra\v cunamo $\vfi^{t+1}_1$, iz ena\v cbe~\eqref{fN} pa
        izra\v cunamo $\vfi^{t+1}_N$.}
    \item{Po \v casu se premaknemo $t \to t+1$ in se vrnemo na prvi korak.}
\end{enumerate}


\end{document}
