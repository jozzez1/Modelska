\documentclass[a4 paper, 12pt]{article}
\usepackage[slovene]{babel}
\usepackage[utf8]{inputenc}
\usepackage[T1]{fontenc}
\usepackage[small, width=0.9\textwidth,labelfont=it]{caption}
\usepackage[pdftex]{graphicx}
\usepackage[usenames, dvipsnames]{xcolor}
\usepackage{amssymb, fullpage, float, pdflscape, subcaption, amsmath, color, hyperref}

\renewcommand{\d}{
	\ensuremath{\mathrm{d}}
}

\newcommand{\e}{
	\ensuremath{\mathrm{e}}
}

\renewcommand{\r}{
	\ensuremath{\mathbf{r}}
}

\hypersetup{
	colorlinks=true,
	linkcolor=black!60!red,
	citecolor=black!60!green,
	urlcolor=black!60!cyan
}

\begin{document}

\begin{center}
\textsc{Modelska analiza II}\\
\textsc{2011/12}\\[0.5cm]
\textbf{10. naloga -- Direktno re\v sevanje Poissonove ena\v cbe}
\end{center}
\begin{flushright}
\textbf{Jože Zobec}\\
\end{flushright}

\section{Uvod}

V tej nalogi bomo Poissonovo ena\v cbo re\v sevali po zgledu iz Matemati\v cne fizike II, z razvojem v Fourierovo
vrsto. Problemi so tipa
\[
	\nabla^2 u (x,y) = g (x,y).
\]
Prva naloga zahteva, da izra\v cunamo poves, zaradi lastne te\v ze. To je isto, kot \v ce na idealno brezmasno
plo\v s\v co pritiskamo z neko silo $-F_z(x,y)$ v smeri `$z$'. Iz kurza mehanike kontinuov~\cite[str.~84]{podgornik}
se spomnimo, da poves opi\v se ena\v cba
\[
	\Delta^2 \zeta (x,y) = +F(x,y), \quad \Delta = \nabla^2 = \partial_x^2 + \partial_y^2,
\]
kar lahko razcepimo v dve ena\v cbi:
\begin{align}
	\nabla^2 f (x, y) &= F_z (x,y), \notag \\
	\nabla^2 \zeta (x,y) &= f (x,y),
\end{align}
kjer je $F (x,y) \propto \rho (x,y)$, tj. bomo v brezdimenzijski obliki ena\v cili silo, in povr\v sinsko gostoto
mase opne. Za $f(x,y)$ in $\zeta(x,y)$ veljajo (homogeni) Dirichletovi robni pogoji, tj.
\[
	f(x,y)\Big|_{\partial \mathcal{A}} = \zeta(x,y)\Big|_{\partial \mathcal{A}} = 0,
\]
kjer je $\mathcal{A}$ definicijsko obmo\v cje na\v se diferencialne ena\v cbe.

Prednost uporabe Fourierove transformacije je ta, da diferencialne ena\v cbe v dualnem prostoru (Four. prostor)
postanejo algebrajske, ki pa jih la\v zje re\v simo. \v Zal pa take metode lahko uporabimo zgolj na lepih
geometrijah.

K sre\v ci je druga naloga analogna. Poiskati moramo temperaturni profil valja v stacionarnem stanju, kjer sta
osnovni ploskvi na stalni temperaturi $T_1$, pla\v s\v c pa na stalni temperaturi $T_2$. Problem ima cilindri\v cno
rotacijsko simetrijo. Re\v sevati moramo
\[
	\nabla^2 T (x,y) = \alpha \partial_t T(x,y) = 0,
\]
tj. Laplaceovo ena\v cbo, z (nehomogenimi) Dirichletovimi robnimi pogoji.

\section{Rezultati}

\begin{thebibliography}{9}
	\bibitem{podgornik}
		R. Podgornik
		{\em Mehanika kontinuov},
		2007,
		\url{http://www-f1.ijs.si/~rudi/lectures/mk-1.9.pdf}
\end{thebibliography}

\end{document}
