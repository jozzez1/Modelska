\documentclass[a4 paper, 12pt]{article}
\usepackage[slovene]{babel}
\usepackage[utf8]{inputenc}
\usepackage[T1]{fontenc}
\usepackage[small, width=0.7\textwidth,labelfont=it]{caption}
\usepackage[pdftex]{graphicx}
\usepackage[usenames, dvipsnames]{xcolor}
\usepackage{amssymb, fullpage, float, pdflscape, subcaption, amsmath, color, hyperref}

\renewcommand{\d}{
	\ensuremath{\mathrm{d}}
}

\newcommand{\e}{
	\ensuremath{\mathrm{e}}
}

\renewcommand{\r}{
	\ensuremath{\mathbf{r}}
}

\hypersetup{
	colorlinks=true,
	linkcolor=black!60!red,
	citecolor=black!60!green,
	urlcolor=black!60!cyan
}

\begin{document}

\begin{center}
\textsc{Modelska analiza II}\\
\textsc{2011/12}\\[0.5cm]
\textbf{10. naloga -- Direktno re\v sevanje Poissonove ena\v cbe}
\end{center}
\begin{flushright}
\textbf{Jože Zobec}\\
\end{flushright}

\section{Uvod}

V tej nalogi bomo Poissonovo ena\v cbo re\v sevali po zgledu iz Matemati\v cne fizike II, z razvojem v
Fourierovo vrsto. Problemi so tipa
\[
	\nabla^2 u (x,y) = g (x,y).
\]
Prva naloga zahteva, da izra\v cunamo poves, zaradi lastne te\v ze. To je isto, kot \v ce na idealno
brezmasno plo\v s\v co pritiskamo z neko silo $-F_z(x,y)$ v smeri `$z$'. Iz kurza mehanike
kontinuov~\cite[str.~84]{podgornik} se spomnimo, da elasti\v cno deformacijo opi\v se ena\v cba
\[
	K_c\Delta^2 \zeta (x,y) = +F_z(x,y), \quad \Delta = \nabla^2 = \partial_x^2 + \partial_y^2,
\]
kar lahko razcepimo v dve ena\v cbi:
\begin{align}
	\nabla^2 f (x, y) &= F_z (x,y), \notag \\
	K_c\nabla^2 \zeta (x,y) &= f (x,y),
\end{align}
kjer je $K_c$ elasti\v cni modul opne/tanke plo\v s\v ce. Ker tega ne poznamo se bomo zadovoljili z
re\v sitvijo $f(x,y) \approx -K_c \zeta(x,y)$, saj $f(x,y)$ prav tako re\v si poves, vendar ne kot
posledico elasti\v cnosti, ampak kot posledica minimizacije povr\v sine ob zunanji obremenitvi.
V brezdimenzijski obliki bomo rekli kar $F_z(x,y) = \rho(x,y)$, tj. porazdelitev mase opne. Na\v sa
ena\v cbo bo torej
\[
    \nabla^2 \zeta (x,y) = \rho (x,y).
\]
Prednost uporabe Fourierove transformacije je ta, da diferencialne ena\v cbe v dualnem prostoru (Four.
prostor) postanejo algebrajske, ki pa jih la\v zje re\v simo. \v Zal pa take metode lahko uporabimo zgolj
na lepih geometrijah.

K sre\v ci je druga naloga analogna. Poiskati moramo temperaturni profil valja v stacionarnem stanju,
kjer sta osnovni ploskvi na stalni temperaturi $T_1$, pla\v s\v c pa na stalni temperaturi $T_2$. Problem
ima cilindri\v cno rotacijsko simetrijo. Re\v sevati moramo
\[
	\nabla^2 T (x,y) = g(x,y), \qquad g(x,y) = 0.
\]
Formalno je to Laplaceovo ena\v cba, vendar nehomogeni Dirichletovi robni pogoji poskrbijo, da
$g(x,y) \neq 0$ za $(x,y) \in \partial \mathcal{A}$. Tako imamo spet Poissonovo ena\v cbo s homogenimi
Dirichletovimi robnimi pogoji z $g (x,y)$
\[
	g(x,y) = \left\{\begin{array}{c c}
			-N^2, & \text{ko\ } y = 0, \\
			-N^2, & \text{ko\ } y = 1, \\
			0, & \text{sicer.}
		\end{array}
		\right.
\]
Tu smo predpostavili, da je $\mathcal{A} = [0,1] \times [0,1]$ in da ga razre\v zemo na $N$ kosov
v obeh smereh. V obeh primerih se poslu\v zimo (hitre) sinusne transformacije. Vendar jo za valj
lahko uporabimo samo v osi cilidri\v cne simetije: v radialni smeri so lastne re\v sitve namre\v c
Besselove funkcije. Po analogiji za tudi tu v radialni smeri re\v sujemo tridiagonalen sistem,
ki odvod v radialni smeri aproksimira s kon\v cnimi diferencami. Relevantni del Laplaceovega
operatorja se zapi\v se kot
\[
    \nabla^2 = \frac{1}{r} \partial_r + \partial_r^2 + \partial_z^2,
\]
\v ce prvi odvod aproksimiramo s simetri\v cno prvo diferenco, dobimo
\[
    U^m_{\ell - 1}\bigg(1 - \frac{h}{2r}\bigg) + U^m_{\ell + 1}\bigg(1 + \frac{h}{2r}\bigg) +
        U^m_\ell \big(2 - 4\cos(m\pi/N)\big)
\]
Za mnoge prostorske simetrije operatorja $\nabla^2$ nastopajo trigonometri\v cne funkcije med
lastnimi, kar pomeni, da \v ceprav ne moremo narediti FFT v vseh smereh, ga lahko izvedemo
vsaj v eni smeri, preostale pa pa nadomestimo lahko z diferencami. \v Zal pa je FFT omejena le
na diferencialne ena\v cbe nad lepimi domenami $\mathcal{A}$ (tj. njihov rob je odsekoma raven).

\section{Rezultati}

Za izra\v cun sem kombiniral {\tt C++} in {\tt Octave}, za prikaz pa {\tt Gnuplot} in {\tt MathGL},
da bi dosegel optimalno razmerje med hitrostjo ra\v cunanja, izrisovanja in seveda hitrostjo programiranja.

Funkcijo gostote $\rho (x,y)$ sem si zamislil kot
\[
    \rho (x,y) = \left\{\begin{array}{c c}
            1, & \text{ko nismo na kri\v zu}, \\
            1 + \Delta m, & \text{ko smo na kri\v zu},
        \end{array}
        \right.
\]
kjer je "`kri\v z"' lik iz \v seste naloge. Homogena opna torej ustreza primeru $\Delta m = 0$.
Rezultati so bili zadosti dobre \v ze z $N = 120$. Primerjal sem samo 1D FFT in 2D FFT. Ugotovil
sem, da 2D FFT metoda traja skoraj dvakrat dlje za $N = 120$, vendar pa da bolj natan\v cne rezultate,
kar lahko vidimo na sliki~\ref{gr1}. Poglejmo si rezultate za nekaj razli\v cnih $\Delta m$ na slikah~\ref{gr2},
~\ref{gr2},~\ref{gr3},~\ref{gr4},~\ref{gr5},~\ref{gr6},~\ref{gr7},~\ref{gr8} in \ref{gr9}.

\begin{figure}[H]\centering
    \input{napaka.tex}
    \caption{Ujemanje 1D in 2D FFT po decimalkah. Vidimo, da imamo ujemanje med eno in tremi decimalnimi
        mesti, kar ni dovolj dobro.}
    \label{gr1}
\end{figure}

\begin{figure}[H]\centering
    \includegraphics[width=0.7\textwidth]{solution_120_0_1}
    \vspace{-32pt}
    \caption{2D FFT za homogeno opno.}
    \vspace{-18pt}
    \label{gr2}
\end{figure}

\begin{figure}[H]\centering
    \includegraphics[width=0.7\textwidth]{solution_120_0_2}
    \vspace{-32pt}
    \caption{1D FFT za homogeno opno.}
    \vspace{-18pt}
    \label{gr3}
\end{figure}

\begin{figure}[H]\centering
    \includegraphics[width=0.7\textwidth]{solution_120_1_1}
    \vspace{-32pt}
    \caption{2D FFT za opno, katere "`kri\v z"' ima dvakratno maso.}
    \vspace{-18pt}
    \label{gr4}
\end{figure}

\begin{figure}[H]\centering
    \includegraphics[width=0.7\textwidth]{solution_120_1_2}
    \vspace{-32pt}
    \caption{Ista situacija kot na sliki~\ref{gr4}, le da imamo 1D FFT.}
    \vspace{-18pt}
    \label{gr5}
\end{figure}

\begin{figure}[H]\centering
    \includegraphics[width=0.7\textwidth]{solution_120_-05_1}
    \vspace{-32pt}
    \caption{2D FFT za opno, katere "`kri\v z'" tehta pol toliko, kot preostali del.}
    \vspace{-18pt}
    \label{gr6}
\end{figure}

\begin{figure}[H]\centering
    \includegraphics[width=0.7\textwidth]{solution_120_-05_2}
    \vspace{-32pt}
    \caption{1D FFT za opno, katere "`kri\v z'" tehta pol toliko, kot preostali del.}
    \vspace{-18pt}
    \label{gr7}
\end{figure}

\begin{figure}[H]\centering
    \includegraphics[width=0.7\textwidth]{solution_120_3_1}
    \vspace{-32pt}
    \caption{2D FFT za opno, katere "`kri\v z'" tehta trikrat toliko, kot preostanek.}
    \vspace{-18pt}
    \label{gr8}
\end{figure}

\begin{figure}[H]\centering
    \includegraphics[width=0.7\textwidth]{solution_120_3_2}
    \vspace{-32pt}
    \caption{1D FFT primerjava za prej\v snjo sliko (tj. sliko~\ref{gr8}).}
    \vspace{-8pt}
    \label{gr9}
\end{figure}

Ni velike rezlike -- seveda, na enem, oz. dveh decimalnih mestih, bosta barvna grafa za \v clove\v sko
oko nerazlo\v cljiva, kljub temu, da sem jih dodatno opremil z izohipsami.

\subsection{Cilinder}

Kot sem povedal v uvodu, imamo za cilinder na voljo le 1D FFT. Rezultat je na sliki~\ref{gr10}.

\begin{figure}[H]\centering
    \includegraphics[width=0.7\textwidth]{cilinder_120_4_3}
    \vspace{-32pt}
    \caption{1D FFT temperaturni profil cilindra.}
    \vspace{-8pt}
    \label{gr10}
\end{figure}

Za primerjavo si poglejmo, kaj bi se zgodilo, \v ce bi "`valj"' razvili po sinusih -- dobili temperaturni
profil dolge palice s kvadratnim profilom. Na slikah~\ref{gr11} in~\ref{gr12} vidimo kako velika je
razlika glede na valj (tj. sliko~\ref{gr10}).

\begin{figure}[H]\centering
    \includegraphics[width=0.7\textwidth]{solution_120_4_2}
    \vspace{-32pt}
    \caption{1D FFT temperaturni profil dolge palice s kvadratnim profilom.}
    \vspace{-18pt}
    \label{gr11}
\end{figure}

\begin{figure}[H]\centering
    \includegraphics[width=0.7\textwidth]{solution_120_4_2}
    \vspace{-32pt}
    \caption{2D FFT temperaturni profil dolge palice s kvadratnim profilom.}
    \vspace{-18pt}
    \label{gr12}
\end{figure}

\begin{thebibliography}{9}
    \bibitem{NR}
        W. H. Press, S. A. Teukolsky, W. T. Vetterling, B. P. Flannery,
        {\em Numerical Recipes: The Art of Scientific Computing},
        Cambridge University Press,
        2007
    \bibitem{podgornik}
        R. Podgornik,
        {\em Mehanika kontinuov},
        2007,
        \url{http://www-f1.ijs.si/~rudi/lectures/mk-1.9.pdf}
\end{thebibliography}

\end{document}
