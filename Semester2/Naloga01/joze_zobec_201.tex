\documentclass[a4 paper, 12pt]{article}
\usepackage[slovene]{babel}
\usepackage[utf8]{inputenc}
\usepackage[T1]{fontenc}
\usepackage[small, width=0.9\textwidth,labelfont=it]{caption}
\usepackage[pdftex]{graphicx}
\usepackage{amssymb, fullpage, float, pdflscape, subcaption, amsmath, color}

\newcommand{\e}{
	\ensuremath{\varepsilon}
}

\newcommand{\w}{
	\ensuremath{\omega}
}

\begin{document}

\begin{center}
\textsc{Modelska analiza II}\\
\textsc{2011/12}\\[0.5cm]
\textbf{1. naloga -- Metoda maksimalne entropije}
\end{center}
\begin{flushright}
\textbf{Jože Zobec}\\
\end{flushright}

\section{Uvod}

V tej nalogi se bomo posvetili prilagajanju modelske funkcije na frekven\v cni spekter. Poznamo \v casovno
odvisen odziv. Prek diskretne Fourierove transformacije dobimo frekven\v cni spekter. Zaradi eksperimentalnih
in numeri\v cnih defektov je ta spekter za\v sumljen. Na\v s cilj je zmanj\v sati \v sum in oja\v cati signal.
To storimo s prilagajanjem modelske funkcije na spekter. Prilagajanje ne poteka po metodi najmanj\v sih kvadratov,
je pa z njo intimno povezana. To je tako imenovana avtoregresijska metoda~\cite[str.~302]{sirca}.

Avtoregresijski model je rekurzijska zveza
\begin{equation}
	x(t) = x_t = - \sum_{i = 1}^p a_i x_{t - i} + \e_t,
	\label{avtoregresija}
\end{equation}
kjer je $x_t$ \v casovno odvisen signal, $\e_t$ pa \v sum. Koeficienti $a_i$ so v splo\v snem lahko kompleksni.
Koeficiente $a_i$ lahko dolo\v cimo na razli\v cne na\v cine -- enoli\v cno jih izra\v cunamo glede na avtokorelacijsko
ali kovarian\v cno matriko z Levinson-Durbinovo metodo, lahko pa jih dolo\v cimo tudi z maksimiziranjem entropije po
Burgovi metodi~\cite{sirca}.

Spekter dobimo prek karakteristi\v cnega polinoma avtoregresijskega modela, $p_\mathrm{AR}$:
\begin{equation}
	p_\mathrm{AR}(w) = w^p + \sum_{i = 1}^p a_i w^{p-i} = \prod_{i = 1}^p (w - z_i),
\end{equation}
kjer so $z_i$ ni\v cle tega polinoma, ki jih imenujemo tudi resonance avtoregresijskega modela. Spektralno gostoto
dobimo kot
\begin{equation}
	P (\w) = \frac{\sigma^2_\e}{|p_\mathrm{AR}\big(\exp(2\pi i\nu/N)\big)|^2}, \quad \sigma^2_\e = \text{konst.},
	\label{spekter}
\end{equation}
izpeljava je v~\cite[str. 308-9]{sirca}.
Predpostavki sta, da je spekter stacionaren in da je $\e_t$ beli \v sum z varianco $\sigma^2_\e$. Ker predpostavimo,
da je $\e_t$ beli \v sum z maksimalno varianco se ta metoda imenuje metoda maksimalne entropije (ni ista metoda
maksimalne entropije, s katero izra\v cunamo koeficiente $a_i$). Vse "`dobre"' resonance (torej tiste, ki nosijo
informacijo), so znotraj enotske kro\v znice, resonance, ki pa predstavljajo \v sum, so zunaj enotske kro\v znice. Kadar
imamo vse koeficiente znotraj enotske kro\v znice, pravimo da je na\v s model stabilen.

Burgov algoritem vrne koeficiente stabilnega modela, tj. so vse resonance znotraj enotskega kroga
\v ze po sami konstrukciji metode. V primeru, da je katera izmed resonanc ve\v cja od 1, jo popravimo
$z_i \to \hat{z_i} = z_i/|z_i|$~\cite[str. 305]{sirca}, pri \v cemer s tem ne spremenimo faze.

Glavna prednost avtoregresijskih modelov je ta, da za izra\v cun spektra ne potrebujemo Fourierove transformacije,
ki ima $\mathcal{O}(N \log N)$, ampak zado\v s\v ca zgolj izra\v cun avtokorelacijskih koeficientov, ki imajo
zahtevnost $\mathcal{O}(p^2)$. Ker je $p \ll N$ se to izpla\v ca, saj pove\v cini $p < 40$ popolnoma
zado\v s\v ca~\cite[str. 309]{sirca}. Poleg tega s tem lahko poi\v s\v cemo glavne resonance (ki nastopajo v
konjugiranih parih) in zaradi rekurzijske narave koeficientov, tudi napovemo \v casovni potek, ki bo sledil.
To naredimo kar po ena\v cbi~\eqref{avtoregresija}, tj.
\begin{equation}
	x_{t+1} = -\sum_{i = 1}^p a_i x_{t - i + 1} \pm \sigma_\e,
	\label{napoved}
\end{equation}
kjer pa \v zal napake ne poznamo, lahko pa jo ocenimo z varianco od prej, tj. $\sigma^2_\e$.

\section{Rezultati}
Za izra\v cun ${a_i}$ sem uporabil sem svobodno razli\v cico iz paketa {\tt tsa}, ki spada v {\tt Octave Forge},
natan\v cneje funkcijo {\tt lattice}.

Izra\v cunal sem tudi $\sigma_\e^2$. To koli\v cino sem uporabil za Wienerjev filter in tudi za avtoregresijske
spektre, kot je v en.~\eqref{spekter}.

Rezultati so na slikah~\ref{gr1}, \ref{gr2} in \ref{gr3}.

Preveril sem tudi, \v ce so poli postavljeni znotraj enotskega kroga. Burgova metoda koeficiente $a_i$ res dolo\v ci
tako, da so vse ni\v cle $z_i$ znotraj enotske kro\v znice. Primer za {\tt co2.dat} je na sliki~\ref{graf0}.

\begin{figure}[H]\centering
	\includegraphics[width=12cm, keepaspectratio=1]{co2-poli}
	\caption{Ni\v cle karakteristi\v cnega polinoma na kompleksni ravnini (tj. abscisa predstavlja realno komponento,
		ordinata pa imaginarno). Graf je narejen za primer $p = 36$, {\tt co2-detrend.dat}. Iz argumenta tega
		kompleksnega \v stevila lahko preberemo, kateri resonanci slike~\ref{gr3} ustreza. Poli, ki so
		najblji\v zji enotski kro\v znici so na\v se resonance.}
	\label{graf0}
\end{figure}

\begin{figure}[H]\centering
	\input{val2-spectra.tex}
	\caption{Vidimo, da je $p = 10$ premalo, saj resonanc ne dose\v ze dovolj natan\v cno, prav tako pa so tudi
		preve\v c razmazane. Na spodnji sliki vidimo, da je \v sum avtoregresijskega spektra bistveno manj\v si,
		vendar so zato tudi vrhovi resonanc na ta ra\v cun ni\v zji.Pravi spekter je pomno\v zen z Wienerjevim
		filtrom.}
	\label{gr1}
\end{figure}

\begin{figure}[H]\centering
	\input{val3-spectra.tex}
	\caption{V tem primeru je $p = 10$ bolj\v si za opis \v suma, vendar \v se vedno ne more dobro opisati
		vrhov. Po drugi strani $p = 30$ in $p = 40$ dobro opi\v seta vrhove, vendar dobimo defekte in
		nastane umeten \v sum. Ker so vrhovi pomembnej\v si od \v suma, je smiselno dati prednost rezultatom,
		kjer $p \geq 20$. Pravi spekter je pomno\v zen z Wienerjevim filtrom.}
	\label{gr2}
\end{figure}

\begin{figure}[H]\centering
	\input{co2-detrend-spectra.tex}
	\caption{Za odstranitev trenda sem uporabljal razli\v cne polinome. Najbolje se je obnesel polinom petega
		reda. Kot vidimo, so rezultati za $p = 30$ solidni in primerljivi s sliko~\ref{gr1}. Pravi spekter je
		pomno\v zen v Wienerjevim filtrom. Kro\v zna rekvenca $\w$ je enaka argumentu kompleksnega
		\v stevila, pri katerem dobimo pol na sliki~\ref{graf0}.}
	\label{gr3}
\end{figure}

Sode\v c po teh rezultatih je te\v zko re\v ci, kaj je bolje. Odvisno od situacije. \v Ce \v zelimo zgolj poiskati
resonance je to verjetno najbolje narediti z avtoregresijo, \v ce pa nas zanima spekter je za to verjetno bolje
narediti z eksaktno Fourierjevo transformacijo. Prav tako, \v ce bi radi manj za\v sumljene podatke je spet
odvisno od situacije. \v Ce je \v stevilo resonanc majhno se spla\v ca avtoregresija, sicer pa Wienerjev filter.

\subsection{Linearna napoved}

Po ena\v cbi~\eqref{napoved} uporabimo prvo polovico signala za podatke, na podlagi katerih izra\v cunamo avtoregresijske
koeficiente, drugo polovico pa za kontrolo. Napako sem ocenil iz variance \v suma, $\sigma_\e^2$, katere oceno sem
dolo\v cil s pomo\v cjo Burgove metode. Opazil sem, da vsem grafom amplitude pojenjajo. To je zato, ker za vse pole
velja $|z_i| < 1$, tj. ni nobenega, ki bi bil $|z_i| = 1$.

\begin{figure}[H]\centering
	\input{val2-predict.tex}
	\caption{Na za\v cetku so rezultati v fazi in znotraj pri\v cakovane napake, vendar se \v ze po kratkem
		\v casu amplituda zmanj\v sa in pademo iz faze. Znotraj napake ostanemo ves \v cas \v sele pri
		$p = 136$.}
	\label{pred1}
\end{figure}

\begin{figure}[H]\centering
	\input{val3-predict.tex}
	\caption{Tu je signal dlje \v casa znotran pri\v cakovane napake. Tudi v fazi smo dlje. To je verjetno zato,
		ker spektralna gostota signala manj za\v sumljena.}
	\label{pred2}
\end{figure}

\begin{figure}[H]\centering
	\input{co2-detrend-predict.tex}
	\caption{Tu napoved ostane najdlje, vendar amplituda kar hitro preneha slediti spremembam.}
	\label{pred3}
\end{figure}

\begin{figure}[H]\centering
	\input{luna-RA-predict.tex}
	\caption{Tu napoved ostane najdlje, vendar amplituda kar hitro preneha slediti spremembam.}
	\label{pred4}
\end{figure}

\begin{figure}[H]\centering
	\input{luna-T-predict.tex}
	\caption{Tu napoved ostane najdlje, vendar amplituda kar hitro preneha slediti spremembam.}
	\label{pred5}
\end{figure}

\begin{figure}[H]\centering
	\input{borza-predict.tex}
	\caption{Tu napoved ostane najdlje, vendar amplituda kar hitro preneha slediti spremembam.}
	\label{pred6}
\end{figure}



\begin{thebibliography}{9}
	\bibitem{sirca}
		S. \v Sirca in M. Horvat,
		{\em Ra\v cunske metode za fizike},
		DMFA Zalo\v zni\v stvo,
		(2010)
\end{thebibliography}


\end{document}
