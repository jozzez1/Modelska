\documentclass[a4 paper, 12pt]{article}
\usepackage[slovene]{babel}
\usepackage[utf8]{inputenc}
\usepackage[T1]{fontenc}
\usepackage[small]{caption}
\usepackage[pdftex]{graphicx}
\usepackage{amssymb, fullpage, float, pdflscape, subcaption, amsmath, color}

\newcommand{\e}{
	\ensuremath{\varepsilon}
}

\newcommand{\w}{
	\ensuremath{\omega}
}

\begin{document}

\begin{center}
\textsc{Modelska analiza II}\\
\textsc{2011/12}\\[0.5cm]
\textbf{1. naloga -- Metoda maksimalne entropije}
\end{center}
\begin{flushright}
\textbf{Jože Zobec}\\
\end{flushright}

\section{Uvod}

V tej nalogi se bomo posvetili prilagajanju modelske funkcije na frekven\v cni spekter. Poznamo \v casovno
odvisen odziv. Prek diskretne Fourierove transformacije dobimo frekven\v cni spekter. Zaradi eksperimentalnih
in numeri\v cnih defektov je ta spekter za\v sumljen. Na\v s cilj je zmanj\v sati \v sum in oja\v cati signal.
To storimo s prilagajanjem modelske funkcije na spekter. Prilagajanje ne poteka po metodi najmanj\v sih kvadratov,
je pa z njo intimno povezana. To je tako imenovana avtoregresijska metoda~\cite[str.~302]{sirca}.

Avtoregresijski model je rekurzijska zveza
\[
	x(t) = x_t = - \sum_{i = 1}^p a_i x_{t - i} + \e_t,
\]
kjer je $x_t$ \v casovno odvisen signal, $\e_t$ pa \v sum. Koeficienti $a_i$ so v splo\v snem lahko kompleksni.
Koeficiente $a_i$ lahko dolo\v cimo na razli\v cne na\v cine -- enoli\v cno jih izra\v cunamo glede na avtokorelacijsko
ali kovarian\v cno matriko z Levinson-Durbinovo metodo, lahko pa jih dolo\v cimo tudi z maksimiziranjem entropije z
Burgovo metodo~\cite{sirca}.

Spekter dobimo prek karakteristi\v cnega polinoma avtoregresijskega modela, $p_\mathrm{AR}$:
\[
	p_\mathrm{AR}(x) = x^p + \sum_{i = 1}^p a_i x^{p-i} = \prod_{i = 1}^{p+1} (x - z_i),
\]
kjer so $z_i$ ni\v cle tega polinoma, ki jih imenujemo tudi resonance avtoregresijskega modela. Spektralno gostoto
dobimo kot
\[
	P (\w) = \frac{\sigma^2_\e}{|p_\mathrm{AR}\big(\exp(2\pi i\w/N)\big)|^2}, \quad \sigma^2_\e = \text{konst.},
\]
izpeljava je v~\cite[str. 308-9]{sirca}.
Predpostavki sta, da je spekter stacionaren in da je $\e_t$ beli \v sum z varianco $\sigma^2_\e$. Ker predpostavimo,
da je $\e_t$ beli \v sum z maksimalno varianco se ta metoda imenuje metoda maksimalne entropije (ni ista metoda
maksimalne entropije, s katero izra\v cunamo koeficiente $a_i$). Vse "`dobre"' resonance (torej tiste, ki nosijo
informacijo), so znotraj enotske kro\v znice, resonance, ki pa predstavljajo \v sum, so zunaj enotske kro\v znice. Kadar
imamo vse koeficiente znotraj enotske kro\v znice, pravimo da je na\v s model stabilen.

Burgov algoritem vrne koeficiente stabilnega modela, tj. so vse resonance znotraj enotskega kroga
\v ze po sami konstrukciji metode. V primeru, da je katera izmed resonanc ve\v cja od 1, jo popravimo
$z_i \to \hat{z_i} = z_i/|z_i|$~\cite[str. 305]{sirca}, pri \v cemer s tem ne spremenimo faze.

Glavna prednost avtoregresijskih modelov je ta, da za izra\v cun spektra ne potrebujemo Fourierove transformacije,
ki ima $\mathcal{O}(N \log N)$, ampak zado\v s\v ca zgolj izra\v cun avtokorelacijskih koeficientov, ki imajo
zahtevnost $\mathcal{O}(p^2)$. Ker je $p \ll N$ se to izpla\v ca, saj pove\v cini $p < 40$ popolnoma
zado\v s\v ca~\cite[str. 309]{sirca}. Poleg tega s tem lahko poi\v s\v cemo glavne resonance (ki nastopajo v
konjugiranih parih) in zaradi rekurzijske narave, tudi napovemo \v casovni potek, ki bo sledil.

\section{Rezultati}
Implementacije Burgove metode ima aplikacijo razli\v cnih numeri\v cnih orodjih -- {\tt Matlab}, {\tt GSL} \ldots
Uporabil sem svobodno razli\v cico iz paketa {\tt tsa}, ki spada v {\tt Octave Forge}, natan\v cneje funkcijo {\tt lattice}.
Ta poleg tega, da dolo\v ci optimalne parametre tudi poi\v s\v ce optimalen red (kateraga zgornja in spodnja meja
je prepu\v s\v cena uporabniku)

\begin{thebibliography}{9}
	\bibitem{sirca}
		S. \v Sirca in M. Horvat,
		{\em Ra\v cunske metode za fizike},
		DMFA Zalo\v zni\v stvo,
		(2010)
\end{thebibliography}


\end{document}
