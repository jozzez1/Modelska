\documentclass[a4 paper, 12pt]{article}
\usepackage[slovene]{babel}
\usepackage[utf8]{inputenc}
\usepackage[T1]{fontenc}
\usepackage[small, width=0.7\textwidth,labelfont=it]{caption}
\usepackage[pdftex]{graphicx}
\usepackage[usenames, dvipsnames]{xcolor}
\usepackage{amssymb, fullpage, float, pdflscape, subcaption, amsmath, color, hyperref}

\hypersetup{
	colorlinks=true,
	linkcolor=black!60!red,
	citecolor=black!60!green,
	urlcolor=black!60!cyan
}

\newcommand{\vfi}{
    \ensuremath{\varphi}
}

\renewcommand{\d}{
    \ensuremath{\delta}
}

\newcommand{\der}[3][]{
    \ensuremath{ \frac{\partial^{#1} #2}{\partial #3^{#1}} }
}

\renewcommand{\Re}{
    \ensuremath{\mathrm{Re}}
}

\newcommand{\sfrac}[2]{
    \ensuremath{\textstyle{\frac{#1}{#2}}}
}

\newcommand{\z}{
    \ensuremath{\zeta}
}

\newcommand{\V}{
    \ensuremath{\mathbf{v}}
}

\begin{document}

\begin{center}
\textsc{Modelska analiza II}\\
\textsc{2011/12}\\[0.5cm]
\textbf{12. naloga -- Navier-Stokesov sistem}
\end{center}
\begin{flushright}
\textbf{Jože Zobec}\\
\end{flushright}

\section{Razmislek}

V tej nalogi bomo re\v sevali Navier-Stokesov sistem dvo-dimenzionalne nestisljive
teko\v cine (kapljevine). Izmed mo\v znosti, ki so nam na voljo, bomo za to vajo
uporabili metodo, ki reducira eno izmed ena\v cb -- metodo, ki dinamiko napove prek
vrtin\v cnosti, `$\zeta$',
\[
    \der{\zeta}{t} + \der{u\zeta}{x} + \der{v\zeta}{y} - \frac{1}{\Re}\bigg(\der[2]{\zeta}{x}
        + \der[2]{\zeta}{y}\bigg) = 0.
\]
Hitrostno polje, $\V = (u, v)$, dobimo iz tokovne funkcije, `$\psi$', 
\[
    u = \der{\psi}{y}, \qquad v = -\der{\psi}{x},
\]
ki je z vrtin\v cnostjo povezana prek Poissonove ena\v cbe,
\begin{equation}
    \nabla^2 \psi = \zeta.
    \label{poisson}
\end{equation}
Pri tem je zagotovljena identiteta $\partial_x u + \partial_y v = 0$. \v Casovni korak je omejen
po Courtanovem pogoju. Robni pogoj za $\psi$ prevedemo na robni pogoj $\zeta$, toka v steno ne
sme biti. Vse odvode prepi\v semo v diskretne in dobimo eksplicitne ena\v cbe za \v casovni razvoj:
\begin{equation}
    \z_{i,j}^{t+1} = \z_{i,j}^{t} + \Delta_{i,j}^t,
\end{equation}
\begin{align}
    \Delta_{i,j}^t =& \sfrac{\d}{h}\Big[\sfrac{1}{h\Re}\big(\z^t_{i,j+1} + \z^t_{i,j-1} + \z^t_{i+1,j} +
        \z^t_{i-1,j} - 4\z^t_{i,j}\big) \notag \\
        &- \sfrac{1}{2}\big((u\z)^t_{i,j+1} - (u\z)^t_{i,j-1} + (v\z)^t_{i+1,j} - (v\z)^t_{i-1,j}\big)\Big],
\end{align}
hitrostno polje dobimo kot
\begin{equation}
    \sfrac{1}{2h}\big(\psi^t_{i+1,j} - \psi^t_{i-1,j}\big) = u^t_{i,j}, \quad
    \sfrac{1}{2h}\big(\psi^t_{i,j-1} - \psi^t_{i,j+1}\big) = v^t_{i,j}.
\end{equation}
Ena\v cbo~\eqref{poisson} lahko re\v sujemo na mnogo razli\v cnih na\v cinov, vendar mislim,
da gre najhitreje z metodo {\tt SOR} s \v Cebi\v sevim pospe\v sevanjem konvergence.

\section{Implementacija}

Tako kot prej sem napisal karseda za hitrost optimizirano razli\v cico programa v programskem jeziku
{\tt C}, kjer sem najbolj po\v zre\v sne funkcije pohitril/nadomestil ob namigih progrma {\tt gprof}.
Dodatno optimizacijo sem napravil tako, da je za mojo $N \times N$ mre\v zo bil $N$ vedno lih. Sku\v sal sem
napraviti program, ki bi \v cim manj preverjal parametre in se \v cim bolj osredoto\v cil na ra\v cunanje.

Nekako je bilo treba pravilno dolo\v ziti $\delta$ (tj. \v casovni korak). Tega sem dolo\v cil tako, da sem
prvih 10 \v casovnih iteracij $\delta$ ra\v cunal sproti s hitrostmi in rekel
\[
    \delta = \frac{1}{40 N v_\text{max}},
\]
nato pa preostanek upo\v steval po zadnjem izra\v cunanem $\delta$. Za izra\v cun nove iteracije $\zeta$ sem
imel pripravjeno pomo\v zen blok spomina. Za izra\v cun $\zeta^{t+1}$ sem ju zamenjal v konstantem \v casu
(tj. zgolj zamenjal naslov spominskega bloka).

\begin{table}[H]\centering
    \caption{Spodaj vidimo kolik\v sen dele\v z \v casa izvajanja programa vzame posamezen del algoritma.
    Kot vidimo, se je \v Cebi\v sevo pospe\v sevanje izpla\v calo, saj je ra\v cunanje Poissonove ena\v cbe
    primerljivo z ra\v cunanjem eksplicitne \v casovne sheme za $\zeta$.}
    \begin{tabular}{r|l}
        dele\v z & Proces \\
        \hline
        $31 \%$ & lihi koraki ({\tt SOR}) \\
        $31 \%$ & sodi koraki ({\tt SOR}) \\
        $22 \%$ & $\zeta^{t} \to \zeta^{t+1}$ \\
        $11 \%$ & izra\v cuni $u_{ij}$ in $v_{ij}$ \\
        $0 \%$  & izra\v cun $\delta$ \\
        $0 \%$  & zamenjava spominskega bloka \\
        \hline
        $5 \%$  & ostalo (alokacija spomina, robni pogoji $\zeta$ \ldots)
    \end{tabular}
\end{table}

Za\v cetno stanje sem izbral $\psi_{i,j} \equiv 0$, $v_{i,j} \equiv 0$. Tudi $u_{i,j}$ je povsod 0, razen na
spodnjem robu, tam je $u_{N,j} \equiv 1$. Zaradi tega je tudi $\zeta_{i,j}$ povsod 0, razen vrstici 
$\zeta_{N-1,j}$ in $\zeta_{N,j}$.

\begin{figure}[H]\centering
    \includegraphics[width=0.8\textwidth]{Re-0}
    \caption{Tokovnice za $\Re = 0.1$, $N = 101$.}
    \label{gr1}
\end{figure}

\begin{figure}[H]\centering
    \includegraphics[width=0.8\textwidth]{Re-100}
    \caption{Tokovnice za $\Re = 100$, $N = 101$.}
    \label{gr2}
\end{figure}

\begin{figure}[H]\centering
    \includegraphics[width=0.8\textwidth]{Re-1000}
    \caption{Tokovnice za $\Re = 1000$, $N = 101$.}
    \label{gr3}
\end{figure}
\end{document}
