\documentclass[a4 paper, 12pt]{article}
\usepackage[slovene]{babel}
\usepackage[utf8]{inputenc}
\usepackage[T1]{fontenc}
\usepackage[small, width=0.7\textwidth,labelfont=it]{caption}
\usepackage[pdftex]{graphicx}
\usepackage[usenames, dvipsnames]{xcolor}
\usepackage{amssymb, fullpage, float, pdflscape, subcaption, amsmath, color, hyperref}

\hypersetup{
	colorlinks=true,
	linkcolor=black!60!red,
	citecolor=black!60!green,
	urlcolor=black!60!cyan
}

\newcommand{\vfi}{
    \ensuremath{\varphi}
}

\renewcommand{\d}{
    \ensuremath{\delta}
}

\newcommand{\der}[3][]{
    \ensuremath{ \frac{\partial^{#1} #2}{\partial #3^{#1}} }
}

\renewcommand{\Re}{
    \ensuremath{\mathrm{Re}}
}

\newcommand{\sfrac}[2]{
    \ensuremath{\textstyle{\frac{#1}{#2}}}
}

\newcommand{\z}{
    \ensuremath{\zeta}
}

\newcommand{\V}{
    \ensuremath{\mathbf{v}}
}

\begin{document}

\begin{center}
\textsc{Modelska analiza II}\\
\textsc{2011/12}\\[0.5cm]
\textbf{12. naloga -- Navier-Stokesov sistem}
\end{center}
\begin{flushright}
\textbf{Jože Zobec}\\
\end{flushright}

\section{Razmislek}

V tej nalogi bomo re\v sevali Navier-Stokesov sistem dvo-dimenzionalne nestisljive
teko\v cine (kapljevine). Izmed mo\v znosti, ki so nam na voljo, bomo za to vajo
uporabili metodo, ki reducira eno izmed ena\v cb -- metodo, ki dinamiko napove prek
vrtin\v cnosti, `$\zeta$',
\[
    \der{\zeta}{t} + \der{u\zeta}{x} + \der{v\zeta}{y} - \frac{1}{\Re}\bigg(\der[2]{\zeta}{x}
        + \der[2]{\zeta}{y}\bigg) = 0.
\]
Hitrostno polje, $\V = (u, v)$, dobimo iz tokovne funkcije, `$\psi$', 
\[
    u = \der{\psi}{y}, \qquad v = -\der{\psi}{x},
\]
ki je z vrtin\v cnostjo povezana prek Poissonove ena\v cbe,
\begin{equation}
    \nabla^2 \psi = \zeta.
    \label{poisson}
\end{equation}
Pri tem je zagotovljena identiteta $\partial_x u + \partial_y v = 0$. \v Casovni korak je omejen
po Courtanovem pogoju. Robni pogoj za $\psi$ prevedemo na robni pogoj $\zeta$, toka v steno ne
sme biti. Vse odvode prepi\v semo v diskretne in dobimo eksplicitne ena\v cbe za \v casovni razvoj:
\begin{equation}
    \z_{i,j}^{t+1} = \z_{i,j}^{t} + \Delta_{i,j}^t,
\end{equation}
\begin{align}
    \Delta_{i,j}^t =& \sfrac{\d}{h}\Big[\sfrac{1}{h\Re}\big(\z^t_{i,j+1} + \z^t_{i,j-1} + \z^t_{i+1,j} +
        \z^t_{i-1,j} - 4\z^t_{i,j}\big) \notag \\
        &- \sfrac{1}{2}\big((u\z)^t_{i,j+1} - (u\z)^t_{i,j-1} + (v\z)^t_{i+1,j} - (v\z)^t_{i-1,j}\big)\Big],
\end{align}
hitrostno polje dobimo kot
\begin{equation}
    \sfrac{1}{2h}\big(\psi^t_{i+1,j} - \psi^t_{i-1,j}\big) = u^t_{i,j}, \quad
    \sfrac{1}{2h}\big(\psi^t_{i,j-1} - \psi^t_{i,j+1}\big) = v^t_{i,j}.
\end{equation}
Ena\v cbo~\eqref{poisson} lahko re\v sujemo na mnogo razli\v cnih na\v cinov, vendar mislim,
da bo \v slo najhitreje z metodo {\tt SOR} s \v Cebi\v sevim pospe\v sevanjem konvergence.

\end{document}
