\documentclass[a4 paper, 12pt]{article}
\usepackage[slovene]{babel}
\usepackage[utf8]{inputenc}
\usepackage[T1]{fontenc}
\usepackage[small, width=0.9\textwidth,labelfont=it]{caption}
\usepackage[pdftex]{graphicx}
\usepackage{amssymb, fullpage, float, pdflscape, subcaption, amsmath, color, hyperref}

\renewcommand{\d}{
	\ensuremath{\mathrm{d}}
}

\newcommand{\e}{
	\ensuremath{\mathrm{e}}
}

\begin{document}

\begin{center}
\textsc{Modelska analiza II}\\
\textsc{2011/12}\\[0.5cm]
\textbf{7. naloga -- Metoda kon\v cnih elementov: Poissonova ena\v cba}
\end{center}
\begin{flushright}
\textbf{Jože Zobec}\\
\end{flushright}

\section{Uvod}

Spet se vra\v camo k Poissonovi ena\v cbi, ki se v brezdim. obliki zapi\v se kot
\begin{equation}
	\nabla^2 v = -1.
\end{equation}
Prostor diskretiziramo tako, da ga najprej tlakujemo s poljubnimi liki. Najprikladnej\v si so
trikotniki, saj lahko z njmi relativno natan\v cno opi\v semo vsako obliko.
Za to je posebej prikladna Delaunayeva triangulacija, ki ima implementacije v raznoterih
knji\v znicah in programih. Uporabil sem preprost odprtokoden program \texttt{Triangle},
za katerega je dovolj, da le podamo robove, s trikotniki jih pa zapolni sam in jih po potrebi
tudi izri\v se. Napisan je v jeziku {\tt C}, ima nadvse prijazen vmesnik (za razliko od knji\v znice
{\tt GCAL}, ta program razumemo v 10 min, \v ce smo vsaj malo vajeni dela z ukazno vrstico)
in je primeren tudi za skriptiranje v konzolah.

\begin{thebibliography}{9}
	\bibitem{sirca}
		S. \v Sirca in M. Horvat,
		{\em Ra\v cunske metode za fizike},
		DMFA Zalo\v zni\v stvo,
		(2010)
	\bibitem{triangle}
		{\tt http://www.cs.cmu.edu/\~{}quake/triangle.html}
\end{thebibliography}

\end{document}
