\documentclass[a4 paper, 12pt]{article}
\usepackage[slovene]{babel}
\usepackage[utf8]{inputenc}
\usepackage[T1]{fontenc}
\usepackage[small, width=0.8\textwidth]{caption}
\usepackage[pdftex]{graphicx}
\usepackage{amssymb, amsmath, fullpage, float, hyperref, color}

\newcommand{\diag}{
	\operatorname{diag}
}

\begin{document}

\begin{center}
\textsc{Modelska analiza I}\\
\textsc{2011/12}\\[0.5cm]
\textbf{6. naloga -- Var\v cni modeli}
\end{center}
\begin{flushright}
\textbf{Jo\v ze Zobec}
\end{flushright}

\section{Uvod}

Teoreti\v cne parametre se v fiziki iz eksperimentov (slednji so lahko tudi numeri\v cni) ponavadi prebere iz
prilagajanja modelske krivulje na merske to\v cke. Krivuljo prilagajamo tako, da za dolo\v cen nabor parametrov
zadostimo arbitrarnim (po navadi minimizacijskim) zahtevam. Najbolj popularen kriterij je ti. $\chi^2$, ki parametre
minimizira tako, da je vsota napak za posamezne merske to\v cke \v cim manj\v sa. Tudi za prilagajanje imamo na
voljo dovolj \v sirok nabor metod, med najbolj znanimi je linearna regresija, katero bomo uporabili v tej
nalogi.

Linearna regresije prek $\chi^2$ sku\v sa prilagoditi vektor parametrov $\beta$ tako, da bo ena\v cba $y = X\beta +
\varepsilon$ imela \v cim manj\v si vektor napak $\epsilon$. To naredimo prek minimizacije $\chi^2$, ki je kar
kvadratna forma
\begin{equation}
	\chi^2 = 2J = (y - X\beta)^T R^{-1} (y - X\beta).
	\label{chi}
\end{equation}
Zahteva, da je $\chi^2$ minimalen vrne, da so idealni parametri
\begin{equation}
	\beta = (X^T R^{-1} X)^{-1} X^T R^{-1} y, \qquad P = (X^T R^{-1} X)^{-1}.
	\label{regres}
\end{equation}
kjer je $\beta$ vektor parametrov na\v sega modela, $X$ je strukturna matrika, ki $\beta$ povezuje z $y$, matrika $R$
je kovarian\v cna matrika merskih to\v ck (kadar so slednje nekorelirane je to diagonalna matrika kvadratov napak
posameznih to\v ck), matrika $P$ pa je kovarian\v cna matrika optimalnih parametrov $\beta$. Ena\v cbi~\eqref{chi}
in~\eqref{regres} sta primer splo\v sne linearne regresije (zato, ker dopu\v s\v camo, da so merske to\v cke lahko
korelirane in vsaka pri druga\v cni napaki -- tj. matrika $R$ ima lahko karseda splo\v sno obliko.

Kaj pa, kadar nam teorija ne napoveduje oblike krivulje? V takem primeru enostavno posku\v samo, vendar se vsej ko
prej lahko zgodi, da bo na\v sa modelska krivulja imela parametre, ki nimajo fizikalnega pomena -- red na\v sega
modela je prevelik. Ho\v cemo seveda, da bo na\v s model karseda var\v cen -- tj. imel \v cim manj parametrov.
To dose\v zemo s pomo\v cjo singularnega razcepa matrike $X$. Singularne vrednosti, ki so zelo majhne,
namigujejo na to, da imamo preve\v c prostih parametrov v modelu in da niso vsi parametri linearno neodvisni.

\section{Razmislek}

V prvem primeru (tj. toplotna prevodnost) prilagajamo testno "`krivuljo"' $\lambda(T, P)$ in sku\v samo poiskati
tako, ki bo imela najmanj parametrov in najbolj\v si $\chi^2$. Na\v s model bo najve\v c tretjega reda, tj.
\[
	\lambda = \sum_{i,j = 0}^3 \beta_{ij} T^i P^j.
\]
Parametrov $\beta_{ij}$ je najve\v c 10, nekateri izmed teh parametrov bodo ni\v c (oz. jih zavr\v zemo), torej
lahko strukturno matriko $X$ ra\v cunamo le enkrat (ta matrika je dimenzije $n \times 10$, $n$ je \v stevilo merskih
to\v ck) nato pa zavra\v camo stolpce. Najve\v c lahko zavr\v zemo 9 stolpcev (\v ce zavr\v zemo vseh 9, potem
nima smisla), najmanj pa nobenega, torej dobimo $G$ na\v cinov, kako lahko to storimo:
\[
	G = \sum_{k = 0}^{9} \binom{10}{k} = 1023.
\]
To ni tako veliko. V tej nalogi bom sku\v saj izra\v cunati vse kombinacije (to\v ck je malo in $\sim 10^3$ mno\v zenj
majnih matrik ni tako potratno) in potem nadaljeval z najbolj zanimivimi.

Sedaj pridemo k drugemu problemu -- kako poiskati oz. predstaviti in klasificirati vse mo\v zne kombinacije? To je
\v cisto preprosto. Vsako izbiro lahko predstavimo z binarnim \v stevilom -- vsak stolpec iz matrike $X$ bodisi izberemo,
bodisi zavr\v zemo. Ker je teh stolpcev $10$ dobimo dvoji\v sko \v stevilo dol\v zine $10$. Od tod tudi pride, da je
\v stevio kombinacij $1023$, saj
\[
	G = 2^n - 1,
\]
kjer `$-1$' pride zato, ker kombinacije $(00\ldots0)$ ne upo\v stevamo, saj ta nima smisla. Vendar ta
dvoji\v ska \v stevila lahko enoli\v cno pretvorimo nazaj v deseti\v ska in dobimo indeks izbire $X$, ki je ravno med $1$
in $1023$. Ta preslikava je bijektivna, tj. smo s tem posredno poiskali vse mo\v zne kombinacije.

Kadar uporabimo $\chi^2$ za kvantitativno mero ujemanja na\v sega modela danimi eksperimentalnimi to\v ckami, je predvsem
prikladno uporabiti $\overline{\chi^2}$, tj. reduciran $\chi^2$, ki je definiran kot
\begin{equation}
	\overline{\chi^2} \equiv \frac{\chi^2}{n - k - 1},
\end{equation}
kjer je $n$ \v stevilo merskih to\v ck, $k$ pa je \v stevilo parametrov $\beta_{ij}$ ki so razli\v cni od $0$ (spet torej
koliko stolpcev smo sprejeli). Koli\v cina $\overline{\chi^2}$ ima predvsem to lepo lastnost, da je za krivulje, ki se
bolje ujemajo s to\v ckami, bli\v zje $1$.

Ko bomo izra\v cunali za vse mo\v zne primere izbire $X$ vse mo\v zne $\chi^2$, bomo izlu\v s\v cili primere, ki imajo
$\overline{\chi^2} \sim 1$ in napravili dodatno analizo, tj. preverili \v stevilo prostih parametrov krivulje in
napravili singularni razcep matrike $X$ ter pod drobnogled vzeli kovarian\v cno matriko $P$.

\section{Rezultati}

Tabela~\ref{tab1} prikazuje toplotno prevodnost $\lambda$ jekla Armco v odvisnosti od temperature in mo\v ci grelca.
Temperaturo kot mo\v c bi lahko reskalirali tako, da bi bila med $0$ in $1$, vendar tu niti ni take potrebe
(brezdimenzijsko transformacijo se po navadi uvaja v sisteme takrat, kadar ho\v cemo dobiti re\v sitev neodvisno od
skale problema, te pa v tem primeru ni).

\begin{table}[H]\centering
	\caption{Tabela meritev za toplotno prevodnost jekla Armco.}
	\begin{tabular}{r|c|c|l}
		$T [^\circ\mathrm{F}]$ & $P[\mathrm{W}]$ & $\lambda[\mathrm{Btu/h\ ft\ {}^\circ F}]$ & $\sigma_{\lambda}$ \\
		\hline
		90  & 276 & 42.345  & 0.28 \\
		100 & 545 & 41.60   & 0.16 \\
		149 & 275 & 39.5375 & 0.28 \\
		161 & 602 & 37.7875 & 0.16 \\
		206 & 274 & 37.3525 & 0.28 \\
		227 & 538 & 36.4975 & 0.16 \\
		247 & 274 & 36.36   & 0.28 \\
		270 & 550 & 35.785  & 0.16 \\
		352 & 272 & 33.915  & 0.28 \\
		362 & 522 & 34.53   & 0.16
	\end{tabular}
	\label{tab1}
\end{table}

Imamo $10$ merskih to\v ck, torej imamo najve\v c $10$ prostih parametrov za na\v s model, kar je v skladu s prej\v snjim
razmislekom. Stolpci maksimalne matrike $X$ so
\[
	X = \Big[1\ |\ T\ |\ P\ |\ T^2\ |\ TP\ |\ P^2\ |\ T^3\ |\ T^2P\ |\ TP^2\ |\ P^3 \Big].
\]
Konfiguracija $(1001001000)$ predstavlja da v $X$ obdr\v zimo konstanto, $T^2$ in $T^3$, tj. da za model vzamemo
$\lambda \approx \beta_{00} + \beta_{20} T^2 + \beta_{30} T^3$ in je o\v stevil\v cena z indeksom $2^9 + 2^6 + 2^3 = 584$.
Matrika $R$ je diagonalna, $R = \diag(\sigma_{\lambda_1}^2, \sigma_{\lambda_2}^2, \ldots, \sigma_{\lambda_{10}}^2)$,
vektor $y$ pa vsebuje $\lambda_i$, kjer je $i$ indeks meritve. Sedaj imamo vse kar potrebujemo za izra\v cun.

\begin{figure}[H]\centering
	\input{prelim.tex}
	\caption{Konstantni \v clen vstopi v konfiguraciji $512$. Vidimo, da lahko tudi brez konstantnega \v clena dobimo
		dobre rezultate, vendar so redki. S konstantnim \v clenom je $\overline{\chi^2}$ takoj ni\v zji za
		velikostni red.}
\end{figure}
Izmed vseh teh mo\v znosti sem se nato odlo\v cilo, da bom zavrgel vse, za katere je $\overline{\chi^2} < 0.5$,
$\overline{\chi^2} > 8$, $k > 7$.

Model z najmanj\v sim $k$, ki ustreza temu kriteriju ima $k = 3$ in nosi indeks $i = 832$. Model je
$\lambda \approx \beta_{00} + \beta_{10}T + \beta_{20} T^2$ in je na sliki~\ref{najman}.
\begin{figure}[H]\centering
	\includegraphics[width=14cm, keepaspectratio=1]{832.png}
	\caption{Odvisnost od $P$ v tem primeru zanemarimo, in upo\v stevamo paraboli\v cno odvisnost od $T$.}
	\label{najman}
\end{figure}
Temu soroden je model $833$, ki predpostavlja $\lambda \approx \beta_{00} + \beta_{10}T + \beta_{20} T^2 + \beta_{03} P^3$.
Vidimo ga na sliki~\ref{sestra}.
\begin{figure}[H]\centering
	\includegraphics[width=14cm, keepaspectratio=1]{833.png}
	\caption{V tem modelu imamo splo\v sno paraboli\v cno odvisnost od temperature in strogo kubi\v cno od mo\v ci grelca.
		Ujemanje je kar solidno, vendar se ni za prehvaliti.}
	\label{sestra}
\end{figure}

Graf, ki ima $\overline{\chi^2}$ najblji\v zje $1$, je $i = 567$ s $\overline{\chi^2} = 1.35$. Tu imamo model
$\lambda \approx \beta_{00} + \beta_{11} TP + \beta_{02} P^2 + \beta_{21} T^2P + \beta_{12} TP^2 + \beta_{03} P^3$,
vendar ima eno izmed singularnih vrednosti precej majhno. Singularne vrednosti $X$ za ta model so
$S = \diag(4.19 \cdot 10^8, 8.56 \cdot 10^7, 1.25 \cdot 10^7, 8.48 \cdot 10^4, 1.82 \cdot 10^4, 2.86 \cdot 10^{-1})$.
Rezultat je na sliki~\ref{najmanjsi}.
\begin{figure}[H]\centering
	\includegraphics[width=14cm, keepaspectratio=1]{567.png}
	\caption{\v Ceprav ima zelo dobro ujemanje v $\overline{\chi^2}$ ta graf mogo\v ce ni najbolj\v si sode\v c
		po singularnih vrednostih.}
	\label{najmanjsi}
\end{figure}

Grafov, ki so ustrezajo prej\v snjemu kriteriju je $152$, kar je preve\v c. Uvedel sem nov pogoj: vse singularne
vrednosti morajo biti ve\v cje od $1$. To nam vrne 18 grafov, od katerih so vsi (zanimivo) brez konstantnega \v clena
in imajo bodisi $5$, bodisi $6$ elementov. Te konfiguracije so v tabeli~\ref{tab2}.
\begin{table}[H]\centering
	\caption{Tu vidimo modele, ki so bili na koncu sprejeti. Zadnji stolpec predstavlja najmanj\v so singularno
		vrednost.}
	\begin{tabular}{r|c|c|l}
		konf. & $\overline{\chi^2}$ & $k$ & $s_{min}$ \\
		\hline\hline
		183 & 2.36 & 6 & 40.15 \\
		187 & 2.69 & 6 & 40.66 \\
		217 & 6.48 & 5 & 56.38 \\
		243 & 1.98 & 6 & 38.93 \\
		245 & 7.19 & 6 & 41.96 \\
		407 & 4.38 & 6 & 35.55 \\
		409 & 3.28 & 5 & 46.16 \\
		411 & 4.13 & 6 & 35.04 \\
		413 & 4.05 & 6 & 35.31 \\
		437 & 3.83 & 6 & 35.66 \\
		438 & 1.78 & 6 & 14.95 \\
		441 & 4.17 & 6 & 32.54 \\
		442 & 6.55 & 6 & 16.75 \\
		465 & 2.58 & 5 & 35.23 \\
		467 & 3.17 & 6 & 29.45 \\
		469 & 3.15 & 6 & 32.08 \\
		473 & 2.02 & 6 & 7.81 \\
		497 & 3.21 & 6 & 27.64
	\end{tabular}
	\label{tab2}
\end{table}
Iz tabele~\ref{tab2} nato izberemo graf, ki ima najmanj parametrov in je najblji\v zlje $\overline{\chi^2} = 1$ in dobimo
konfiguracijo $465$. Ta najbolje zado\v s\v ca kriteriju po \v cim manj\v sem \v stevilu parametrov $k$. Ta model je
\begin{equation}
	\lambda = \beta_{10} T + \beta_{01} P + \beta_{20} T^2 - \beta_{02} P^2 + \beta_{03} P^3,
\end{equation}
parametri z napakami (ki smo jih prebrali iz kovarian\v cne matrike) so v tabeli~\ref{tab3}, graf pa je prikazan
na sliki~\ref{zmaga}
\begin{table}[H]\centering
	\caption{Parametri najbolj ugodnega modela.}
	\begin{tabular}{r|c|l}
		             & $\beta_{ij}$& $\sigma_{\beta_{ij}}$ \\
		\hline
		$\beta_{10}$ & $-6.3 \cdot 10^{-2}$ & $\pm 3.9 \cdot 10^{-3}$ \\
		$\beta_{01}$ & $3.2 \cdot 10^{-1}$  & $\pm 4.5 \cdot 10^{-3}$ \\
		$\beta_{20}$ & $7.3 \cdot 10^{-5}$  & $\pm 8.6 \cdot 10^{-6}$ \\
		$\beta_{02}$ & $-6.7 \cdot 10^{-4}$ & $\pm 1.9 \cdot 10^{-5}$ \\
		$\beta_{03}$ & $4.3 \cdot 10^{-7}$  & $\pm 2.0 \cdot 10^{-8}$
	\end{tabular}
	\label{tab3}
\end{table}
Matrika singularnih vrednosti $X$ je
\begin{equation}
	S = \diag(3.85 \cdot 10^8,\ 1.85 \cdot 10^5,\ 6.16 \cdot 10^4,\ 89.57,\ 35.23).
\end{equation}

\begin{figure}[H]\centering
	\includegraphics[width=14cm, keepaspectratio=1]{465.png}
	\caption{To\v ckam prilagojen polinomski nastavek za $\lambda (T, P)$. To je na\v sa zadnja izbira, ki je dovolj
		nizka v $\overline{\chi^2}$, ima vse singularne vrednosti ve\v cje od $10$ in hkrati najmanj koeficientov
		v razvoju.}
	\label{zmaga}
\end{figure}

\end{document}

